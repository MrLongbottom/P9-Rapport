% This file demonstrates how to use the IEEEConf LaTeX2e macro package,
% to prepare a manuscript for proceedings on CD of the conference
% FedCSIS
%
\documentclass[a4paper,10pt]{IEEEtran}
%\documentclass[a4paper]{IEEEconf}



% Usepackages
%%%%%%%%%%%%%%%%%%%%%%%%
%Article stuff
%%%%%%%%%%%%%%%%%%%%%%%%
% This package serves to balance the column lengths on the last page of the document.
% please, insert \balance command in the left column of the last page
\usepackage{balance}
\usepackage{supertabular}
\usepackage{paralist}

%% to enable \thank command
\IEEEoverridecommandlockouts 
%% The usage of the following packages is recommended
%% to insert graphics
\usepackage[dvips]{graphicx}
% to typeset algorithms
\usepackage{algorithm}
\usepackage{algpseudocode}
% to typeset code fragments
\usepackage{listings}
% to make an accent \k be available
\usepackage[T1]{fontenc}
% provides various features to facilitate writing math formulas and to improve the typographical quality of their output.
\usepackage[cmex10]{amsmath}
\interdisplaylinepenalty=2500
% por urls typesetting and breaking
\usepackage{url}
% for vertical merging table cells
\usepackage{multirow}
\usepackage{siunitx}


%%%%%%%%%%%%%%%%%%%%%%%%
%Mixed packages
%%%%%%%%%%%%%%%%%%%%%%%%
\usepackage[hidelinks]{hyperref}
\usepackage{graphicx}
\usepackage[textsize=scriptsize]{todonotes}
\usepackage{glossaries}
\usepackage[numbers]{natbib}
\usepackage{subcaption}
\bibliographystyle{plainnat}
\usepackage{amssymb}
\let\labelindent\relax %Needed before enumitem to avoid error with IEEE template
\usepackage{enumitem}
% Encoding %
\usepackage[utf8]{inputenc}
\usepackage{enumitem}
\usepackage{framed}


%%%%%%%%%%%%%%%%%%%%%%%%
%Figures
%%%%%%%%%%%%%%%%%%%%%%%%

\usepackage{tikz}
\usepackage{pgfplots}

% Tables %
\usepackage{tabularx}
\usepackage{xltabular}
\newcolumntype{Y}{>{\centering\arraybackslash}X}
\usepackage{booktabs} % for professional tables
\usepackage{tikz}
\usetikzlibrary{positioning}
\usepackage{longtable}
\usepackage{makecell}
\usetikzlibrary{shapes.geometric, arrows}
\usetikzlibrary{matrix,positioning,arrows.meta,arrows}

\tikzset{my arrow/.style={-latex},
	set@com@col/.style={},set@com@col@aryarg/.style={column #1/.style={set@com@col}},
	set@com@row/.style={},set@com@row@aryarg/.style={row #1/.style={set@com@row}},
	set common column/.style 2 args={set@com@col/.style={#2}, set@com@col@aryarg/.list={#1}},
	set common row/.style 2 args={set@com@row/.style={#2}, set@com@row@aryarg/.list={#1}},
}

%%%%%%%%%%%%%%%%%%%%%%%%
%Defines
%%%%%%%%%%%%%%%%%%%%%%%%

\newtheorem{definition}{\textbf{Definition}}

% Autoref
\def\sectionautorefname{Section}
\def\subsectionautorefname{Section}
\def\subsubsectionautorefname{Section}
\def\figureautorefname{Fig.}
\def\definitionautorefname{Definition}


% Macros
%%%%%%%%%%%%%%%%%%%%- Commands -%%%%%%%%%%%%%%%%%%%%%%%%%%%%%
\newcommand{\langballe}[2][]{\todo[color = cyan, #1]{\textbf{Langballe:} #2}}
\newcommand{\vejleder}[2][]{\todo[color = magenta, #1]{\textbf{Vejleder:} #2}}
\newcommand{\simba}[2][]{\todo[color = green, #1]{\textbf{Simba:} #2}}
\newcommand{\dennis}[2][]{\todo[color = pink, #1]{\textbf{Dennis:} #2}}


% Glossaries
\newacronym{lda}{LDA}{latent Dirichlet allocation}
\newacronym{NLP}{NLP}{Natural Language Processing}
\newacronym{tf-idf}{tf-idf}{Term Frequency-Inverse Document Frequency}
\newacronym{lm}{LM}{Language Model}
\newacronym{vi}{VI}{Variational Inference}
\newacronym{pr}{PR}{PageRank}
\newacronym{ppr}{PPR}{Personalized PageRank}
\newacronym{bm25}{BM25}{Okapi Best Matching 25}
\newacronym{ir}{IR}{Information Retrieval}
\newacronym{map}{MAP}{Mean Average Precision}

%Glossary changes
%%%%%%%%%%%%%%%%%%%%%%%%
% Switch off hyperlinks for all uses of \gls etc.
% Hyperlinks will be inserted manually in the custom display style
\setkeys{glslink}{hyper=false}

\renewcommand*{\CustomAcronymFields}{%
	name={\the\glsshorttok},%
	description={\the\glslongtok},%
}

\renewcommand*{\SetCustomDisplayStyle}[1]{%
	\defglsentryfmt[#1]{%
		\ifdefempty\glscustomtext
		{%
			\ifglsused\glslabel
			{% subsequent use
				% Assuming all acronyms are written in upper case, so
				% not bother to check for case changes.
				\glsifplural
				{% subsequent use, plural
					\glshyperlink[\glsentryshortpl{\glslabel}]{\glslabel}%
				}%
				{% subsequent use, singular
					\glshyperlink[\glsentryshort{\glslabel}]{\glslabel}%
				}%
			}%
			{% first use
				\glsifplural
				{% first use, plural
					\glscapscase
					{% no case change
						\glstarget{\glslabel}{\glsentrylongpl{\glslabel}\glsinsert}%
						\space(\glsentryshortpl{\glslabel})%
					}%
					{% first letter upper case
						\glstarget{\glslabel}{\Glsentrylongpl{\glslabel}\glsinsert}%
						\space(\glsentryshortpl{\glslabel})%
					}%
					{% all caps
						\glstarget{\glslabel}{\MakeTextUppercase{%
								\glsentrylongpl{\glslabel}\glsinsert}}%
						\MakeTextUppercase{\space(\glsentryshortpl{\glslabel})}%
					}%
				}%
				{% first use, singular
					\glscapscase
					{% no case change
						\glstarget{\glslabel}{\glsentrylong{\glslabel}\glsinsert}%
						\space(\glsentryshort{\glslabel})%
					}%
					{% first letter upper case
						\glstarget{\glslabel}{\Glsentrylong{\glslabel}\glsinsert}%
						\space(\glsentryshort{\glslabel})%
					}%
					{% all caps
						\glstarget{\glslabel}{\MakeTextUppercase{%
								\glsentrylong{\glslabel}\glsinsert}}%
						\MakeTextUppercase{\space(\glsentryshort{\glslabel})}%
					}%
				}%
			}%
		}%
		{% \glsdisp used
			\ifglsused\glslabel
			{% subsequent use
				\glshyperlink[\glscustomtext]{\glslabel}%
			}%
			{% first use
				\glstarget{\glslabel}{\glscustomtext}%
			}%
		}%
	}%
}

\SetCustomStyle


%
%
\title{Improving Search using a combination of Random Walk and Topic Modeling}
%
%
\author{
\IEEEauthorblockN{Dennis Højbjerg Rose, Peter Langballe Erichsen, and Rasmus Engesgaard Christensen}\\
\IEEEauthorblockA{Department of Computer Science, Aalborg University,\\Selma Lagerløfs Vej 300, 9220 Aalborg Øst, Denmark\\Email: \{drose16, perich16, rech16\}@student.aau.dk}}

% conference papers do not typically use \thanks and this command
% is locked out in conference mode. If really needed, such as for
% the acknowledgment of grants, issue a \IEEEoverridecommandlockouts
% after \documentclass

% for over three affiliations, or if they all won't fit within the width
% of the page, use this alternative format:
% 
%\author{\IEEEauthorblockN{Michael Shell\IEEEauthorrefmark{1},
%Homer Simpson\IEEEauthorrefmark{2},
%James Kirk\IEEEauthorrefmark{3}, 
%Montgomery Scott\IEEEauthorrefmark{3} and
%Eldon Tyrell\IEEEauthorrefmark{4}}
%\IEEEauthorblockA{\IEEEauthorrefmark{1}School of Electrical and Computer Engineering\\
%Georgia Institute of Technology,
%Atlanta, Georgia 30332--0250\\ Email: see http://www.michaelshell.org/contact.html}
%\IEEEauthorblockA{\IEEEauthorrefmark{2}Twentieth Century Fox, Springfield, USA\\
%Email: homer@thesimpsons.com}
%\IEEEauthorblockA{\IEEEauthorrefmark{3}Starfleet Academy, San Francisco, California 96678-2391\\
%Telephone: (800) 555--1212, Fax: (888) 555--1212}
%\IEEEauthorblockA{\IEEEauthorrefmark{4}Tyrell Inc., 123 Replicant Street, Los Angeles, California 90210--4321}}

% \usepackage{showframe}
\begin{document}
\maketitle              % typeset the title of the contribution

% - Abstract (5-20 lines)
\begin{abstract}
	We investigate whether using a topic model can improve \gls{ir} methods.
We employ a probabilistic topic model, specifically \gls{lda}, to automatically extract topics from Danish news articles.
%How can \gls{pr} be used on a dataset without explicit edges?
We use the topic distribution, from the \gls{lda}, to construct the adjacency matrix edge by measuring the Jensen-Shannon distance between each article topic distribution, since references between news articles do not exist.
We use \gls{lda}, \gls{tf-idf}, \gls{bm25}, \gls{lm} and \gls{pr} as our baselines for testing \gls{ir} performance.
We also combine these in various ways to investigate the individual performance of each method.
We evaluate the \gls{ir} methods by generating document and topic queries which are based on the topic distribution from the \gls{lda}.
\gls{bm25} is the best performing baseline, we have, and we improve \gls{bm25} by combining it with \gls{pr}.
\end{abstract}
\glsresetall
%\input{sections/Simba.tex}
%  Keywords
%\input{sections/Keywords.tex}
%
%\glsresetall
%
%% - Introduction (7.5 \%) 
\section{Introduction}
%% Argue briefly for the relevance of the studied area/problem — start from general and end with your concrete problem.
A general challenge for search engines, no matter the environment, is how to acquire relevant search results from a given corpus.
While some search engines use simple algorithms, such as finding all articles including a given search term and sorting by date, this may not always give useful results.
Because of this we want to explore ways to improve search results while using information already available from a corpus.

A way to improve searches could be by labeling documents in a corpus and using these labels to get results.
A popular way to label documents is topic modeling.
Topic modeling gives each document in a corpus a distribution of topics, which can then be used for different applications.
More specifically, we look into \gls{lda} since it is currently the most used topic model, and is applicable for many situations\cite{lda}.

Topic modeling does not by itself help with how to rank documents in a search, which is why an additional part is necessary.
We have chosen to explore a popular way to rank documents, which is by using PageRank.
PageRank is an algorithm originally meant for ranking web pages in search engine results by their importance, but it can easily be extended to other types of data, such as news articles.
A more specific implementation of PageRank is a cluster-based PageRank\cite{ClusterPageRank}.
This method uses the similarities between documents, clusters, and a corpus as an extension to regular PageRank.
We believe this can help find further connections between documents by clustering them using the topics given by \gls{lda}.

Our goal is thus to improve news article search by using a combination of topic modeling and PageRank.
For our experiments we use the Nordjyske database of news articles as our dataset.
The idea is to use topic modeling to apply topics to documents, and cluster the documents with similar topics.
Then the cluster-based PageRank, which takes these clusters into account, is used in order to find relevant documents given a search query.

%A combination of LDA along with cluster PageRank for generating relevant search results
\todo[inline]{Contributions}

The paper is organized as follows:
In \autoref{sec:related-works}, articles related to topic modeling and random walk are investigated.
\autoref{sec:dataset} describes the dataset used in the experiments.
In \autoref{sec:method}, the steps of our method are shown, and \autoref{sec:prepro}, \autoref{sec:language_model}, \autoref{sec:lda}, \autoref{sec:search_query}, and \autoref{sec:pagerank} describe these steps further.
In \autoref{sec:experiment}, the test combinations and hyperparameters are detailed and results are shown.
\autoref{sec:discussion} analyzes and discusses these results.
Finally, in \autoref{sec:conclusion} and \autoref{sec:future_work}, conclusions and future work are given.

%
%% - Related works (7.5 \%))
\section{Related Work}\label{sec:related-works} \vejleder[inline]{add some more}
In this section, we investigate existing literature concerning \gls{lda} and similar topic models, as well as relevant work related to random walk.

\citet{lda} describes a generative statistical model, which is heavily used within topic modeling. 
The intuition behind the \gls{lda} topic model is that a document has an underlying distribution of topics, and that the words of a document are based on these topics.
This model is used widely within the field of topic modeling and is somewhat considered a standard, see section \autoref{sec:lda} for more detail.

There also exists a large amount of extensions to \gls{lda}.
\citet{blei2012topicmodels} describes some of the most significant extensions that are made from relaxing the assumptions of \gls{lda}.
Some extensions mentioned are: the dynamic topic model\cite{blei2006dynamic} which assumes that topics change over time, and thus respects the order of the documents, and the correlated topic model\cite{blei2007correlated} and pachinko allocation model\cite{li2006pachinko} which each allow correlations to exist between topics.


\citet{quanti} describes a method of how to quantitatively preprocess and analyze large amounts of journalistic data dating from 1945-2000. 
They propose \gls{lda} as a way of categorizing articles and using this as the basis for search.
However, they only check for a single subject, namely the nuclear topic.  


\citet{Tang2008} successfully combine the two research fields of Topic Modeling and Random Walk search. 
Though their focus is mainly on mapping author and venue relations as part of their topic model, their methods for combining these two fields are still novel and generally applicable.


\citet{yang2009topic} presents a four step method for doing the topic-level random walk to search scientific articles. 
They also describe various combinations of \gls{lda} and PageRank.
These combinations are evaluated against information retrieval baselines such as BM25\cite{bm251996}.


While our method in some ways will be similar to the work of \citeauthor{Tang2008} and \citeauthor{yang2009topic}, our method will have a significant change of working with a new data set consisting of documents without explicit connections between documents, namely news articles.
\todo[inline]{we might need to focus on only applying LM*LDA*PageRank to the new dataset}
This will be detailed further in the following sections.

%Articles, we might want in our related works section:
%\begin{itemize}
%	\item \cite{lda} (written)
%	\item \cite{ClusterPageRank} (written)
%	\item \cite{Tang2008} (written)
%	\item \cite{jelodar2019latent}
%\end{itemize}

%
%% - Preliminaries/"Experiment" (20 \%)
%%		- Data
%%		- Setup
%%		- Analysis
\section{Dataset}\label{sec:dataset}
Nordjyske is a media group maintaining many local newspapers, radios, and websites in the North Jutland region of Denmark.
All articles produced by Nordjyske are saved in a nonpublic database.
The dataset, which we use, consists of $\sim\! 63.000$ articles from 2017.
These documents consist of many attributes where we extract the ID, headline, and body of the article.
They do contain more information, such as authors, references to images, article type, date and time of publishing, etc.
This information could become relevant with the use of various \gls{lda} modifications, however this is beyond the scope of this paper.
Some entries within the dataset are ads or scoreboard, which we try to filter away in the preprocessing phase which is explained in \autoref{sec:prepro}.

An example article can be seen in \autoref{prepro:example1}.

\begin{figure}[h]
	\begin{framed}
		\begin{quote}
			headline: \textit{'Tricktyve fik fingre i dankort HOBRO:'}
			body: \textit{'Frække tricktyve har atter slået til i en af Hobros dagligvareforretninger. Denne gang gik det ud over en 87-årig mand, som under indkøbsturen i Superbrugsen i Adelgade blev frastjålet sit Visa-Dankort med tilhørende pin-kode.'}
		\end{quote}
	\end{framed}
		\caption{Snippet of extracted information of an article in our dataset.}
		\label{prepro:example1}
\end{figure}

\section{Pipeline}

\tikzstyle{process} = [rectangle, rounded corners, minimum width=2cm, minimum height=1cm,text centered, draw=black, fill=gray!50]
\tikzstyle{decision} = [diamond, minimum width=3cm, minimum height=1cm, text centered, draw=black, fill=green!30]

\begin{figure}[ht]
    \centering
    \begin{tikzpicture}[node distance=2cm]
    %\draw[step=1cm,gray,very thin] (-8,-8) grid (8,8);
	\node (Dataset) [process] {Nordjyske Dataset};
	\node (Cleaning)[process, below of=Dataset] {Preprocessing Phase};
	\node (Training) [process, below of=Cleaning] {Train IR Methods};
	\node (Cluster PR) [process, below of=Training] {Evaluate Models};
	\node (Query) at (-4.5, -6) [process] {Query Generation};
	\node (Result) [process, below of=Cluster PR] {Results};
	\draw [->, very thick] (Dataset) edge (Cleaning); 
	\draw [->, very thick] (Cleaning) edge (Training);
	\draw [->, very thick] (Training) edge (Cluster PR);
	\draw [->, very thick] (Cluster PR) edge (Result);
	\draw [->, very thick] (Query) edge (Cluster PR);
\end{tikzpicture}
	\caption{The method visualized as a flowchart, where a dataset consisting of articles is processed into a list of ranked results.}
    \label{fig:process_figure}
\end{figure}

The pipeline of our framework is divided into five phases which are displayed in \autoref{fig:pipeline}.

\subsection{Article dataset}
We have an data set consisting of articles from the media group Nordjyske. Their primary focus is to maintain a variety of local newspapers within the North Jutland region of Denmark. 
The data ranges from 2017-2019, where a total of ~270.000 articles have been extracted from their database.

\subsection{Preprocessing}
The extracted articles use the format:
\begin{itemize}
	\item Id
	\item Headline
	\item Body
\end{itemize}
When training the LDA model, we concatenate the Headline and Body for simplicity.
The preprocessing phase is described further in detail in \todo[inline]{ref til preprocessing}.
This phase applies some processes to simplify the dataset and remove redundant information. 
After the preprocessing phase, we are left with ~130.000 articles.

\subsection{LDA}
We apply the \acrfull{LDA} to the dataset to generate topic clusters based on the content of the articles. 
We describe the investigation and selection of hyper parameters in \todo{ref til LDA section}. 
After the model has been trained, we have a document-topic distribution matrix $\theta$ and a topic-word distribution matrix $\varphi$.
$\theta$ will be used as clusters in the next phase.

\subsection{Clustering PageRank}
In this phase, we apply a clustering random walk\todo{cite clusteirng pagerank} to the articles and clusters given in the \gls{LDA} phase.
The algorithm yields a prioritized list of articles based on the query provided.


\subsection{Query Processing}
The query given to the search algorithm is stemmed and turned into a list of words. 
This list of words are then queried against LDA and to check which cluster each word belongs to.
Lastly it is transformed into a personalization vector to use within the Clustering PageRank.



%
\section{\gls{ir} Methods}\label{sec:ir_methods}

In this section we will be introducing the \gls{ir} methods that we will be testing.
We have chosen these methods to evaluate different ways of performing a \acrfull{ir} task.

The goal of each method is to produce a document-score vector with one score for each document in the corpus, indicating the relevance of the document compared to the query.
These scores can then be sorted in descending order to produce a ranking of the documents.

Some of the methods calculate scores for a single word at a time rather than a whole query.
In that case, we treat their score is an estimate of the probability a document $d$ generating a specific word $w$.
To find the probability of $d$ generating a given query $q$, we take the product of the word probabilities for each word in the query $w \in q$, as the document would have to generate each word, as shown in \autoref{eq:query_prob}.

\begin{equation}\label{eq:query_prob}
	P(q|d) = \prod_{w \in q} P(w|d)
\end{equation}

\section{\acrlong{lda}}\label{sec:lda}
\Gls{lda} is a probabilistic topic model method. 
The objective of topic modeling is to infer topics (collections of words) in a document set.
The result consists of a topic-word distribution matrix $\beta$, which for each topic gives a distribution of words belonging to said topic, and a document-topic distribution matrix $\theta$ which for each document gives a distribution of topics to which the document belongs to.

\citeauthor{lda} \cite{lda} introduce \gls{lda}, which has since become a staple within topic modeling.
\gls{lda} works under the assumption that documents are generated from a specific generative process, and tries to reverse engineer this process.
This generative process assumes that documents are random mixtures of latent topics and that each topic is a distribution over all the words in the corpus.

\begin{definition}\label{def:topic}
	\textit{A topic is a distribution over words in a corpus}
\end{definition}

This process generates $K$ topics and $D$ documents containing $N_{d}$ words, where $d$ is a document in $D$.
The generative process has two Dirichlet distributions $Dir(\alpha)$ for the document-topic relation and $Dir(\eta)$ for the topic-word relation.

\begin{figure}[h]
	\centering
	\includegraphics[width=0.5\textwidth]{figures/Smoothed_LDA.jpg}
	\caption{Plate notation for \gls{lda}. Boxes symbolize repeated processes, shaded elements are observed information. Image is from \url{https://en.wikipedia.org/wiki/Latent_Dirichlet_allocation}}
	\label{fig:lda}
\end{figure}
\vejleder[inline]{beskriv inference og learning af modelen i appendix}
From these Dirichlet distributions, we can sample two multinomial distributions: document-topic $\theta$ and topic-word $\beta$.
These Dirichlet distributions are tuned with the hyperparameters $\alpha$ and $\eta$, which adjust the entropy of the sampled distributions.
An $\alpha$ value near 1 causes each document to be distributed over almost all topics, while an $\alpha$ value near 0 causes each document to be distributed over only a few topics.
Similarly $\eta$ will adjust how many words each topic contains.
This also has the added consequence that a high $\alpha$ will make documents appear more similar, and a high $\eta$ will make topics appear more similar.

Using $\theta$ and $\beta$, we can sample concrete topics $Z$ from documents, and concrete words $W$ from topics.
\autoref{fig:lda} gives an overview of the assumed generative process.

\todo[inline]{mention the limitations of lda.}

\section{Language model}
In \cite{yang2009topic}, they describe various combinations of \gls{lda} and other models. 
The language model they describe is similar to a query likelihood model, which generates a probability of how likely a given document $d$ produces a given query $q$.
To calculate this probability, they need to find the likelihood for each word in the query by using function 

$$ P(w|d) = \frac{N_d}{N_d + \lambda} \cdot \frac{tf(w,d)}{N_d} + (1 - \frac{N_d}{N_d + \lambda}) \cdot \frac{tf(w,D)}{N_D} $$
where $N_d$ is the number of word tokens in $d$ and $tf(w,d)$ is the word frequency of $w$ in $d$. $\lambda$ is a Dirichlet smoothing factor and is set to the average document length.
When $D$ is used, it is referring the document collection.
We multiply the word probabilities together in order to get the probability for $q$.

$$ P(q|d) = \prod_{w \in q} P(w|d) $$
 
The \gls{lda} model gets poor retrieval performance when it not used in combination with another model \cite{yang2009topic}.
They combine the \gls{lda} model and the language model to get information about the topic and word correlation between $q$ and $d$.

\section{PageRank}\label{sec:pagerank}
We construct a graph over our documents, where edges between documents $d_i$ and $d_j$ are based on the similarity between them $sim(d_i, d_j)$.

This allows us to use PageRank on the graph to search for similar documents.
On top of this, we can add a personalization vector, to guide the search in a specific direction.
We can base this personalization vector on a search query to find documents similar to the query.

To construct the adjacency matrix for the graph, we apply a similarity function on each pair of document-topic distributions $\theta_{d_i}, \theta_{d_j}$, from our topic model.

We use Jenson-Shannon distance as our similarity function, which is a measure of the distance between probability distributions\cite{jensen-shannon2003}\cite{jensen-shannondis2003}.
It is calculated using the square root of the Jenson-Shannon divergence between two probability distributions and gives a score from 0 to 1.
However since Jenson-Shannon calculates difference rather than similarity, a low value means more similarity, making our similarity function:
$$sim(d_1, d_2) = 1 - JS(\theta_{d_1}, \theta_{d_2})$$


%\subsection{Clustering PageRank}
%
%In \cite{ClusterPageRank}, they describe the \gls{Cluster-CMRW} which is a new version of the random walk model. 
%They improve upon this model by incorporating information from clusters. 
%The clusters are found by employing three different clustering algorithms.
%They create a new transition probability matrix where the cluster information is added by combining three similarity functions based on the information levels present in the data.
%They describe three similarity levels between:
%\begin{itemize}
%    \item Sentence to Sentence - $f(v_i \rightarrow v_j)$
%    \item Sentence to Topic Cluster - $\omega(v_i, clus(v_i))$
%    \item Topic Cluster to Document Set - $\pi(clus(v_i))$
%\end{itemize}
%where $v_i$ is the given sentence and $clus(v_i)$ is the cluster that $v_i$ belongs to.
%
%We want to use this method to incorporate the information given by the topic distributions within a \gls{Cluster-CMRW}. 
%We first create three similarity functions to incorporate topic cluster level information into a new adjacency matrix.
%\begin{itemize}
%    \item Document to Document - $f(d_i \rightarrow d_j)$
%    \item Document to Topic Cluster - $\omega(d_i,clus(d_i))$
%    \item Topic Cluster to Document Set - $\pi(clus(d_i))$
%\end{itemize}
%
%\noindent
%Where $d$ is a document, $clus(d_i)$ is a topic cluster of document $d_i$.
%Notice that a document $d$ can contain multiple topics, as opposed to normal clustering where each element is generally assumed to only be part of one cluster.
%
%\subsection*{Document to Document Similarity}
%As in \autoref{sec:pagerank} we use Jensen-Shannon distance as our base for document-to-document similarity:
%$$ f(d_i \rightarrow d_j) = 1 - JS(d_i, d_j)$$ 
%
%\subsection*{Document to Topic Cluster Similarity}
%We define the similarity between a document and a topic cluster as the document-topic probability distribution between them.
%$$ \omega(d_i,clus(d_i)) = \theta_{d_i,clus(d_i)}$$
%where $\theta$ is the document-topic distribution matrix.
%
%\subsection*{Topic Cluster to Document Set Similarity}
%We define the similarity between a topic $t$ and the whole document set as:
%$$ \pi(clus(d_i)) = \frac{\sum_{d}^{D} \theta_{clus(d_i)}}{D} $$
%where $D$ is the number of documents in the document set.
%
%
%To create the adjacency matrix, \cite{ClusterPageRank} linearly combine the different similarity functions, in which we intend to do the same.
%Formally, the adjacency matrix is calculated with the following function.
%$$ f(d_i \rightarrow d_j | clus(d_i), clus(d_j)) $$
%which is evaluated to 
%
%\begin{align*}
%&f(d_i \rightarrow d_j | clus(d_i), clus(d_j)) = \\
%&f(d_i, d_j) \cdot (\lambda \cdot \pi(clus(d_i))) \cdot \omega(d_i, clus(d_i)) \\ 
%&+ (1-\lambda) \cdot \pi(clus(d_j)) \cdot \omega(d_j, clus(d_j))
%\end{align*}
%
%where $\lambda$ is a weight between $[0,1]$ that controls the relative contribution of the two clusters.


\subsection{\acrlong{tf-idf}}
\Gls{tf-idf} is a well known \gls{ir} method which is used to find the most important words within text.
As the name suggests, this methods aims to balance the two measures: term frequency and inverse document frequency.
For our case, terms are the words left in the documents after the preprocessing phase.
For each term in the query $t \in q$, a tf-idf score will be calculated for each document in the corpus $d \in D$.
Thus each document will have one score for each term in he query.


Term frequency describes how often a word is used in the specific document that is being evaluated.
Inverse Document Frequency describes how many documents in the corpus includes the given term, with a higher percentage of documents producing a lower value.
The overall goal of \gls{tf-idf}, can be summarized as a measure of how often used and unique a term is for a given document. Words that are often used in the given document, but rarely used in the corpus will have the highest \gls{tf-idf} scores.

The formula for \gls{tf-idf} can be seen in \autoref{eq:tfidf}
\begin{equation}\label{eq:tfidf}
	\text{tf-idf}(w, d, D) = \text{tf}(w, d) \cdot \text{idf}(w, D)
\end{equation}
where tf$(t, d)$ is the number of times term $t$ is in document $d$ and idf$(t, D)$ is the inverse document frequency of $t$ in $D$.

\subsection{\acrlong{bm25}}
\todo[inline]{describe goal / purpose / idea}
\gls{bm25} is a bag of words retrieval function, which is similar to \gls{tf-idf}, since it uses the inverse document frequency.
The formula for \gls{bm25} can be seen in \autoref{eq:bm25}\cite{bm25}.
\begin{equation}\label{eq:bm25}
	\text{bm25}(d, q) = \sum_{i=1}^{n}\text{idf}(q_i) \cdot \frac{\text{tf}(q_i, d) \cdot (k_1 + 1)}{\text{tf}(q_i, d) + k_1 \cdot (1 - b + b \cdot \frac{|d|}{avgdl})}
\end{equation}
where tf$(q_i)$ is the number of times the term $q_i$ appears in $d$.
$|d|$ is the number of words in d. 
$avgdl$ is the average document length from the corpus.
$b$ and $k_1$ are both hyperparameters, which are set to $0.75$ and $1.5$, respectively.

\subsection{Combining \gls{ir} Methods}
Each of the \gls{ir} methods described in this section has their own strengths and weaknesses.
We combine multiple of these methods, to see if they are able to draw upon each others strengths and cover each others weaknesses.
We do with the hope of a combination of methods being able to produce better results than each of the methods separately.

As with \citet{yang2009topic}, we combine multiple \gls{ir} methods together, by normalizing their document-score vectors and then either summing or multiplying them together element-wise.
This process is visualized in \todo{ref figure}.

\todo[inline]{insert figure.}

Both of these two options are viable, and have their own benefits and drawbacks.
Summation allows documents to be ranked fairly high even if one of the combined methods produce a low score.
With multiplication a document with average scores for each method can have a better rank than one with mostly good scores but a really low score from one of the combined \gls{ir} methods.


%\section{Dataset}\label{sec:dataset}
Nordjyske is a media group maintaining many local newspapers, radios, and websites in the North Jutland region of Denmark.
All articles produced by Nordjyske are saved in a nonpublic database.
The dataset, which we use, consists of $\sim\! 63.000$ articles from 2017.
These documents consist of many attributes where we extract the ID, headline, and body of the article.
They do contain more information, such as authors, references to images, article type, date and time of publishing, etc.
This information could become relevant with the use of various \gls{lda} modifications, however this is beyond the scope of this paper.
Some entries within the dataset are ads or scoreboard, which we try to filter away in the preprocessing phase which is explained in \autoref{sec:prepro}.

An example article can be seen in \autoref{prepro:example1}.

\begin{figure}[h]
	\begin{framed}
		\begin{quote}
			headline: \textit{'Tricktyve fik fingre i dankort HOBRO:'}
			body: \textit{'Frække tricktyve har atter slået til i en af Hobros dagligvareforretninger. Denne gang gik det ud over en 87-årig mand, som under indkøbsturen i Superbrugsen i Adelgade blev frastjålet sit Visa-Dankort med tilhørende pin-kode.'}
		\end{quote}
	\end{framed}
		\caption{Snippet of extracted information of an article in our dataset.}
		\label{prepro:example1}
\end{figure}

%%\section{Pipeline}

\tikzstyle{process} = [rectangle, rounded corners, minimum width=2cm, minimum height=1cm,text centered, draw=black, fill=gray!50]
\tikzstyle{decision} = [diamond, minimum width=3cm, minimum height=1cm, text centered, draw=black, fill=green!30]

\begin{figure}[ht]
    \centering
    \begin{tikzpicture}[node distance=2cm]
    %\draw[step=1cm,gray,very thin] (-8,-8) grid (8,8);
	\node (Dataset) [process] {Nordjyske Dataset};
	\node (Cleaning)[process, below of=Dataset] {Preprocessing Phase};
	\node (Training) [process, below of=Cleaning] {Train IR Methods};
	\node (Cluster PR) [process, below of=Training] {Evaluate Models};
	\node (Query) at (-4.5, -6) [process] {Query Generation};
	\node (Result) [process, below of=Cluster PR] {Results};
	\draw [->, very thick] (Dataset) edge (Cleaning); 
	\draw [->, very thick] (Cleaning) edge (Training);
	\draw [->, very thick] (Training) edge (Cluster PR);
	\draw [->, very thick] (Cluster PR) edge (Result);
	\draw [->, very thick] (Query) edge (Cluster PR);
\end{tikzpicture}
	\caption{The method visualized as a flowchart, where a dataset consisting of articles is processed into a list of ranked results.}
    \label{fig:process_figure}
\end{figure}

The pipeline of our framework is divided into five phases which are displayed in \autoref{fig:pipeline}.

\subsection{Article dataset}
We have an data set consisting of articles from the media group Nordjyske. Their primary focus is to maintain a variety of local newspapers within the North Jutland region of Denmark. 
The data ranges from 2017-2019, where a total of ~270.000 articles have been extracted from their database.

\subsection{Preprocessing}
The extracted articles use the format:
\begin{itemize}
	\item Id
	\item Headline
	\item Body
\end{itemize}
When training the LDA model, we concatenate the Headline and Body for simplicity.
The preprocessing phase is described further in detail in \todo[inline]{ref til preprocessing}.
This phase applies some processes to simplify the dataset and remove redundant information. 
After the preprocessing phase, we are left with ~130.000 articles.

\subsection{LDA}
We apply the \acrfull{LDA} to the dataset to generate topic clusters based on the content of the articles. 
We describe the investigation and selection of hyper parameters in \todo{ref til LDA section}. 
After the model has been trained, we have a document-topic distribution matrix $\theta$ and a topic-word distribution matrix $\varphi$.
$\theta$ will be used as clusters in the next phase.

\subsection{Clustering PageRank}
In this phase, we apply a clustering random walk\todo{cite clusteirng pagerank} to the articles and clusters given in the \gls{LDA} phase.
The algorithm yields a prioritized list of articles based on the query provided.


\subsection{Query Processing}
The query given to the search algorithm is stemmed and turned into a list of words. 
This list of words are then queried against LDA and to check which cluster each word belongs to.
Lastly it is transformed into a personalization vector to use within the Clustering PageRank.



\section{Experiments}\label{sec:experiment}

\subsection{Hyperparameters}\label{subsec:hyperparameters}
Before setting up our experiment we did some initial testing in order to find acceptable hyperparameter values for our \gls{lda} model.
% runtime parameters
First we adjusted parameters that had to do with how the algorithms runs to ensure that the model converged, without spending unneccesary resources.
\texttt{passes} adjusts how many times the model passes over the whole corpus, and can be seen as an 'epoch'. This is neccessary for smaller dataset to ensure convergence before the algorithm finishes, but for large enough datasets this parameter can have a value of one, as is the case for \cite{online} where their corpus includes 3.3M documents.
\texttt{iterations} which adjusts the maximum number of iterations of the E-step (from \cite{}, Algorithm 2) is allowed without all documents having converged. Setting this to a high value ensures convergence, but also increases training time. In practice iterations should be set to the lowest possible values where nearly (> 99\%) all documents have converged by the end of trainning.
From testing we found 20 \texttt{passes} and \texttt{iterations} was enough to converge 27678 out of 27919 documents (99.14\%).

% model parameters
Once we had assured that our model would converge, we tested hyperparameters for our \gls{lda} model.
These include:
\begin{itemize}
	\item K - the number of topics
	\item $\alpha$ - dirichlet prior for document-topic distributions
	\item $\eta$ - dirichlet prior for topic-word distributions
\end{itemize}
Similar to clustering algorithms, there is usually an optimal setting for K, where all values above will produce topics which are subtopic of the optimal topics, and all values below will achieve subpar performance.
This theoretical optimal value is however based on the corpus, so will need to be tested in order to find some setting of K that is close to optimal.
$\alpha$ and $\eta$ adjusts the dirichlet distributions that will create the binomial distribtions for document-topic and topic-word relations, respectively.
Lower values will lead to more uneven distributions, favoring fewer topics per document and fewer words per topic, while higher values will make the distribution more uniform distributions.

\begin{table}[h]
	\centering
	\label{tab:params}
	\begin{tabular}{c|c}
		Parameter & Tested Values\\
		\hline
		$K_1$ & 10, 50, 100, 200, 300\\
		$K_2$ & 5, 10, 15, 20, 25, 30, 35, 40, 45, 50\\
		$\alpha$ & 0.5, 0.1, 0.01, 0.001\\
		$\eta$ & 0.1, 0.01, 0.001, 0.0001\\
	\end{tabular}
	\caption{test}
\end{table}

%% - Body(50 \%)
%%	- Results 
%
%%	- Discussion
%%		- Main results
%%		- Other results
%%		- Error sources
\section{Discussion}\label{sec:discussion}
In this section, we analyze and describe the results, we get from running our experiment.

\gls{pr} using our adjacency-matrix successfully improved upon the results of other models in many cases.
This is worth noting as our adjacency matrix was constructed based on similarity of the documents' topic-distributions, rather than any relation embedded in the documents themselves, such as scientific articles referencing each other as with \citeauthor{yang2009topic}\cite{yang2009topic}.
This suggests that other \gls{ir} methods benefit from the similarity context provided by our \gls{pr}, regardless of the validity of the underlying edges.
This means that our initial idea of using the topic distributions as a similarity measure in the \gls{pr} algorithm, was a success.
\todo[inline]{Vurdering af egen adjacency matrix som edges i PR i forhold til original paper. Hvis vi når det: generaliserbarhed af dette (andet datasæt)}

\gls{lm} performed surprisingly bad compared to the results of \cite{yang2009topic}. 
\todo[inline]{We need to figure out what we are going to say about lm's bad performance, or maybe move some of it to appendix if we want to avoid talking too much about it}

\gls{lda}-\gls{ir} gets good topic query retrieval results when combined with \gls{pr}, as seen in \autoref{tab:results}.
This comes at the cost of bad document query retrieval results.
Though \gls{lda}-\gls{ir} has good \gls{map} results, \autoref{tab:results_precision_at_10} shows that \gls{lda}-\gls{ir} only gets the best precision@10 and precision@100 with queries of 4 words.
This means that while it is well suited for ranking a large number of documents fairly well (as with \gls{map}), it is not good at finding a relatively small number of relevant documents
It also points towards \gls{lda}-\gls{ir} being more reliant on a larger amount of context than many of the other models, and could therefore likely benefit greatly from query expansion methods.
\todo{due to?}
However, for this experiment \gls{lda}-\gls{ir}'s purpose is mostly to improve the topic retrieval performance of other models it is combined with, at the cost of reducing their document retrieval performance.

Overall, the most successful combinations of baselines for combining document query retrieval and topic query retrieval seem to be $\gls{bm25}+\gls{pr}$ and $\gls{bm25}*\gls{pr}$.
These combinations get good results in both categories, even managing to outperform the other models in topic query retrieval as measured by precision@10 in \autoref{tab:results_precision_at_10}.
If dealing with larger queries, the combination $\gls{bm25}+\gls{lda}-\gls{ir}+\gls{pr}$ can also be considered as it improves further upon the topic query retrieval; however, this comes at the cost of decreasing the document query retrieval performance further.

%
%
%% - Conclusion (5 \%)
%%	- (Possible) Future work
\section{Conclusion}\label{sec:conclusion}
In this paper we explored the possibility of improving news article search using different combinations of models.
Here the focus has been on exploring whether using the \gls{lda} topic model could improve the performance of the article retrieval.

Now we can answer the following questions from the introduction:
\begin{quote}
	\emph{How can information retrieval techniques be evaluated based on search queries?}
	\begin{itemize}
		\item \emph{How can queries be generated for a dataset?}
		\item \emph{How can evaluation be done in a way that favors abstraction, rather than word frequency?}
	\end{itemize}
\end{quote}
\vspace{0.1 cm}

\begin{quote}
	\emph{How can PageRank be used on a document dataset with no connections between documents?}
\end{quote}

For the first question, queries can be generated in different ways depending on what the purpose of the evaluation is.
We generate two types of queries based on either important words from a specific document or a specific topic.
These are evaluated, respectively, on \acrlong{map} and P@n, to favor both the specificity of finding a specific document and the more abstract idea of finding documents related to a topic.
The results from \autoref{tab:results} and \autoref{tab:results_precision_at_10} indicate that, when searching for documents in a topic, combinations using \gls{lda} can perform better than \gls{bm25} when longer queries are used.

For the second question, we explored the possibility of creating the adjacency matrix for \gls{pr} by using the similarity between \gls{lda} topics, generated for the articles.
From the results in \autoref{tab:results}, \autoref{tab:results_precision_at_10}, and \autoref{tab:hit_results}, this would seem to have been highly effective, since most best performing models include \gls{pr}.
\subsection{Future Work}\label{sec:future_work}

The method, we have presented, can be expanded in different ways. 
An interesting approach, could be to investigate extensions of \gls{lda}.
One such extension could be the Dynamic Topic Model\cite{blei2006dynamic}, which is good at finding specific topics over time, which might improve the retrieval of old and important articles within a large dataset.
This would allow working on more diverse data without having to worry as much about the use of language changing over time.
There is also the approach of \citet{blei2007correlated}, where the similarity between topics are taken into account.
This could make it easier to find appropriate hyperparameter settings as it makes it possible to evaluate how distinct topics in a topic model are.

Another approach to our method is to replace standard \gls{pr} with one taking topics directly into account, like the Topical PageRank\cite{yang2009topic} or \cite{Tang2008}. 
E.g. the transition probability within the adjacency matrix can be changed to focus on the correlation between topics.

%
%\input{sections/Acknowledgements}

% - References (10 \%)
% 	- Acknowledgments
%	- Bibliography
\bibliography{paper}

%Appendixes, if needed, appear before the acknowledgment.
\appendix
\subsection{Project Process}
In this section, we provide a general overview of the project, as well as some of the changes and challenges we underwent during the project.
Below, is a brief overview of the points covered in this section:

\begin{itemize}
	\item Initial Idea - Combine LDA and PageRank
	\item Preprocessing \& LDA implementation
	\item Adjacency matrix construction
	\begin{itemize}
		\item Challenges with calculation time
		\item Focus on more restrictive preprocessing and thresholds
	\end{itemize}
	\item Discovery of related papers
	\item Found out our model did not converge
\end{itemize}

Each of these items will now be described in more detail.

Our initial idea with the project was to combine \gls{lda} with a PageRank algorithm to see if this could improve the \acrlong{ir} results of \gls{lda}.
Originally, we wanted to use the Clustering PageRank method as described in \citet{ClusterPageRank}, but that idea was later scrapped.

We began by implementing some initial preprocessing methods and used an \gls{lda} implementation from Gensim\footnote{\url{https://radimrehurek.com/gensim/models/ldamodel.html}}.

After this, we only needed to combine \gls{lda} with a \gls{pr} implementation. 
To do so, we first needed to construct an adjacency matrix.
Though, when constructing a large adjacency matrix based on a similarity measure, it took an extremely long time, which made us evaluate what changes were necessary to the way we constructed the adjacency matrix and the earlier phases: preprocessing and \gls{lda}.
From this, we made the preprocessing more aggressive and implemented more preprocessing methods, to continually make the dataset smaller while keeping the relevant data.

Importantly, we also implemented thresholds to the topic-document distribution matrix $\theta$ and the topic-word distribution matrix $\beta$ made by the \gls{lda} model.
For each distribution we replaced values below one divided by the number of values in the distribution, with 0.
This was done to make sure that every document wasn't connected to every other document in the adjacency matrix, and to save calculation time by working with sparse matrices.
We later learned that future calculations expected non-zero values, and would be dominated by zeros because of multiplication.
This threshold was therefore removed.

Lastly, we also tried many different ways to construct the adjacency matrix more efficiently, described further in \autoref{app:adj_matrix}.

However, even after we managed to produce an adjacency matrix, our results did not perform as well as expected which made us search for possible solutions.
From here we began searching for more similar papers, where we found: \citet{yang2009topic} and \citet{Tang2008}.
We focused particularly on \cite{yang2009topic}, due to a strong similarity between their project and our ideas.
We used their methods to improve upon several parts of our methods, and we changed from building a single framework, to testing and comparing multiple different methods, as they had done.
However, our results were still not as good as expected.

We then learned through tests of our \gls{lda} model that it was unable to converge using only a single pass through the corpus.
We then began testing hyperparameters for the implementation of \gls{lda}, before once again finding the optimal hyperparameters for our dataset.

\subsection{Encountered Problems}
\todo[inline]{indicate whether the problems were solved.}

\subsubsection{Adjacency Matrix Construction}
Our adjacency matrix is based on whether documents share some topics in their topic distributions. 
This leads us to the problem of $D*D*T$ insertion operations into a sparse matrix with what was originally ~50.000 documents. We never tested how long this would end up taking, but it was definitely infeasible.
We did a lot of preprocessing, which reduced this number eventually all the way down to ~30.000 documents, but it was too insignificant to solve anything. Calculation time was now estimated to be ~140 days.
However our similarity function is symmetric, so we cut the time in half by only constructing half of the adjacency matrix. Calculation time was now estimated to be ~69 days.
We then changed our algorithm to only consider documents that shared topics to the original documents, decreasing our time to $D*T_d*D_t$, where both of the new variables are reduced compared to the originals. This reduced the calculation time to be estimated as ~48 days.
We also made the calculation multithreaded in order to reduce computation time. Estimated calculation time is reduced to ~2,4 days using 8 threads.
Lastly, we changed to construct our matrix using the \emph{lil\_matrix} format, which has faster insertions. This last change made the construction much more manageable.


\todo[inline]{Nedenstående problem skal uddybes, hvis det beholdes. Skal også have sit eget afsnit separat fra Adjacency Matrix Construction}
The function get\_document\_topics might be needing the whole corpus since whenever we give it the query it returns the same two topics. 
These two topics are probably the two most general topics within the model which might spell trouble since we don't know how to remove these from the model.

\subsubsection{Dirichlet distributions only produce non-zero values}
We use document-topic and topic-word distributions produced by the Dirichlet distributions to construct our adjacency matrix and to convert queries into topic distributions. 
However as part of how \gls{lda} is designed it has no zeros in its $\theta$ and $\beta$ values, only extremely low numbers.
This is not optimal for constructing an adjacency matrix, as it would result in a fully connected graph.
Adjusting $\alpha$ and $\eta$ will make the low values lower, and the high values higher, but this does not make the problem disappear.
We have therefore introduced thresholds.
These thresholds reduce all values lower than the threshold to zero in the document-topic matrix and topic-word matrix.
The new problem introduced by introducing thresholds is that the exact values of the thresholds are now added as extra hyper-parameters for our solution.


% Examples of tables and figure
%\input{examples.tex}

%\section{Unused Sections}
%\input{unusedSections/KnowledgeGraph.tex}
\end{document}
