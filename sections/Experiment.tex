\section{Experiments}\label{sec:experiment}

The goal of our experiment is to test the quality of various \gls{ir} models and combinations of \gls{ir} models.

We test various combinations of the following baselines:
\begin{itemize}
	\item \gls{lm}
	\item \gls{lda}
	\item \gls{bm25}
	\item \gls{tf-idf}
	\item \gls{pr}
	\item \gls{ppr}
\end{itemize}
\todo[inline]{clarify the difference between lda and this ir method based on lda. Maybe calling it reverse lda or something?}
\todo[inline]{add a small description of each baseline, especially the ones not described earlier in the paper}

Each baseline will be given a query and based on that query produce values for each document. Results are based on the ranking order of these balues.
However, in this experiment we will not only test the ability to retrieve a specfic document based on a query related to the document, but also the ability to retrieve relevant documents within a specific topic, generated by our topic model.

For each baseline we evaluate the quality of it's \gls{ir} for both queries generated based on documents and queries generated based on topics.
For topic queries, all documents with values higher than the mean value in the topic's vector in the document-topic matrix $\theta_t$ are considered correct retrivals.
This also means that while document queries always only have one ground truth positive, topic queries can have several thousands (~1.000-10.000).
This makes them hard to compare on a single baseline, however it is still possible to compare these two results between multiple baselines.

For evaluation metrics/measures we use \gls{map} and $p @ n$.
Each of these metrics are shown as the mean value for 80 queries of length 1 to 4, making a total of 320 queries.

Specifially we hope to see whether \gls{lda} is able to improve the topic \gls{ir} when combined with other baselines and whether \gls{pr} algorithms operating on a graph build based on topic similarity is able to improve any results when combined with other baselines.

The results of the experiment are shown on \autoref{tab:results}.

\subsection{Results}\label{subsec:results}

\begin{table*}[h]
	\centering
	\caption{Results table}
	\begin{tabular}{l|c|c|c|c|c|c|c|c}
		Model / \gls{map} & D1 & D2 & D3 & D4 & T1 & T2 & T3 & T4 \\
		\midrule
		\gls{lm} & 0.198 & 0.152 & 0.291 & 0.260 & 0.126 & 0.130 & 0.128 & 0.129 \\  
		\gls{lda} & 0.00457 & 0.00527 & 0.0429 & 0.0538 & 0.155 & 0.186 & 0.168 & 0.178 \\
		\gls{bm25} & \textbf{0.270} & 0.656 & 0.866 & \textbf{0.908} & 0.155 & 0.158 & 0.155 & 0.161 \\
		\gls{tf-idf} & 0.210 & 0.621 & 0.799 & 0.897 & 0.155 & 0.157 & 0.155 & 0.161 \\
		\gls{lm} + \gls{lda} & 0.0419 & 0.0214 & 0.0602 & 0.120 & 0.147 & 0.163 & 0.145 & 0.146 \\
		\gls{lm} * \gls{lda} & 0.0931 & 0.0462 & 0.175 & 0.177 & 0.150 & 0.175 & 0.155 & 0.166 \\
		\gls{lm} * \gls{pr} & 0.163 & 0.138 & 0.259 & 0.236 & 0.130 & 0.133 & 0.129 & 0.130 \\
		\gls{lm} + \gls{pr} & 0.170 & 0.153 & 0.283 & 0.256 & 0.130 & 0.132 & 0.130 & 0.131 \\
		\gls{lda} + \gls{pr} & 0.00458 & 0.00526 & 0.0429 & 0.0538 & 0.162 & \textbf{0.195} & \textbf{0.177} & \textbf{0.187} \\
		\gls{lda} * \gls{pr} & 0.00781 & 0.00569 & 0.0410 & 0.0537 & 0.156 & 0.186 & 0.168 & 0.179 \\
		\gls{bm25} + \gls{pr} & 0.269 & 0.656 & 0.866 & 0.902 & \textbf{0.192} & 0.193 & 0.175 & 0.183 \\
		\gls{bm25} * \gls{pr} & 0.267 & \textbf{0.663} & 0.884 & 0.904 & 0.155 & 0.159 & 0.155 & 0.161 \\
		\gls{bm25} + \gls{lda} + \gls{pr} & 0.269 & 0.656 & 0.866 & 0.902 & \textbf{0.192} & 0.193 & 0.175 & 0.183 \\
		\gls{bm25} * \gls{lda} * \gls{pr} & 0.267 & \textbf{0.663} & \textbf{0.884} & 0.904 & 0.155 & 0.159 & 0.155 & 0.161 \\
	\end{tabular}
	
	\label{tab:results}
\end{table*}