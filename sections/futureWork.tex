\subsection{Future Work}\label{sec:future_work}

The method, we have presented, can be expanded in different ways. 
An interesting approach, could be to investigate extensions of \gls{lda}.
One such extension could be the Dynamic Topic Model\cite{blei2006dynamic}, which is good at finding specific topics over time, which might improve the retrieval of old and important articles within a large dataset.
This would allow working on more diverse data without having to worry as much about the use of language changing over time.
There is also the approach of \citeauthor{blei2007correlated}\cite{blei2007correlated}, where the similarity between topics are taken into account.
This could make it easier to find appropriate hyperparameter settings as it makes it possible to evaluate how distinct topics in a topic model are.

Another approach to our method is to replace standard \gls{pr} with one taking topics directly into account, like the Topical PageRank\cite{yang2009topic} or \todo{cite other lda+pr}. 
E.g. the transition probability within the adjacency matrix can be changed to focus on the correlation between topics.
