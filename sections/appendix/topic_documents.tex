\subsection{Representative Documents for Topics}
In \autoref{tab:representative_documents}, 5 documents that are representative of the 5 topics from \autoref{fig:30TopicWords} can be seen.
In this table, the topic number, document number, the weight of the document, and the document itself are shown.
Here, the weight is the probability of the document being generated by a specific topic in the \gls{lda} model.
The weight for a document comes from the topic probability distribution $\theta$ for the document.
The document chosen to represent a given topic is the document with the highest weight for the topic out of all documents.

%Analysis
By examining the content of the representative document for each topic, some deeper knowledge about what the topics cover can be discovered.
The document for topic 0 appears to be an article about the Islamic State (IS), cities and territories related to them, and terrorism.
Topic 0 includes highly weighted words such as "land" (country), "vejr" and "vind" (weather and wind), "Nordkorea" (North Korea), and "is", which might be Islamic State.
These observations seem to indicate that this topic is about countries and news related to this.
The document for topic 1 is about a municipality in Denmark giving 5 million kroner to help build a national park.
Some highly weighted words for this topic are "krone" (Danish currency), "virksomhed" (company), "penge" (money), and "arbejde" (work).
This topic seems to be about money in general, as well as corporations and work, which is not seen in the document.
The document for topic 2 is the least informative of the documents so far.
It is a general article about what events are happening in the near future, with no specific topic.
Since the highly weighted words in the topic also do not seem to have a connection, it might be a topic specifically made of this type of article, which would make the topic not very useful in practice.
It might be worth removing these kinds of articles in the preprocessing, since this could generate a clearer topic.
The document for topic 3 is about the White House appointing a new communications director.
This fits well with the highly weighted words from the topic, such as "hus" (house), "præsident" (president), and "amerikansk" (American), and indicates that it is a topic about the American presidency, and the White House in general.
Finally, the document for topic 4 is a review of different wines.
This shows a clear correlation to the highly weighted words of the topic, such as "god" (good), "pris" (price), "år" (year), and "vin" (wine).
This indicates that topic 4 is mostly about wines and articles related to this.

These observations show that the topics and documents seem to be correctly correlated.
Though there will still appear topics that are not clearly understandably for humans, such as topic 2.

Another observation worth noting is that it is clear by looking at the weights of the documents, that the longer the document is, the higher the probability of being generated by a specific topic is.
This makes sense since the more information the \gls{lda} model has to work with, the more precisely it can place a document in a topic.

%Hvis vi føler tabellen er for stor, kan vi f.eks. ændre det til kun at vise de første 50-100 ord fra hvert dokument
\newpage
\onecolumn
	\begin{xltabular}{\linewidth}{@{}c|c|c|X@{}}
		\toprule
		{\footnotesize Topic} & {\footnotesize \thead{Document\\\#}} & {\footnotesize Weight} & {\footnotesize Document} \\
		\midrule
		0 & 30014 & 0.991 & "selv uden kalifatet vil is ise være en trusselselv uden kalifatet vil is ise være en trussel beirutte først røg ryge røge de ud af millionbyen mosul i irak og nu er jihadisterne fra islamisk stat is ise blive besejre i den syriske by raqqa raqqae der siden side har været hovedsæde for gruppe såkaldt kalifat men selv om is ise har miste stor del dele af sit fysiske territorie så vil der fortsat fortsætte være massere masse af udfordring i syrien og irak vurderer ekspert de advare samtidig om at is ise ideologi vil bestå og at gruppe støtter vil udgøre en terrortrussel i vesten vest den forfærdelig sandhed er at is ise vil være lige så livsfarlig en oprørsgruppe og farligt et terrornetværk som det var da det minded minde om en stat siger mellemøstekspert nicholas nichola heras fellow ved center for a new american security til nyhedsbureau reut udfordring er nu at is ise vil blive et hævngerrigt spøgelse der vil forsøge at skab skabe kaos fortsætter fortsætte han aymenn jawad al tamimi der forsker forske i jihadisme ved tænketank middle east forum er enig han forvente at islamisk stat vil fortsætte med at benytte sig af selvmordsbombere sprængladninger og have såkaldt sove celler rundtom i verden efter kalifatets sammenbrud angribe angreb i europa vil fortsætte noget tid endnu jeg tro at sejr over is ise som et statslignende projekt vil svække gruppe tiltrækningskraft men is ise vil have beirutte først røg ryge røge de ud af millionbyen mosul i irak og nu er jihadisterne fra islamisk stat is ise blive besejre i den syriske by raqqa raqqae der siden side har været hovedsæde for gruppe såkaldt kalifat men selv om is ise har miste stor del dele af sit fysiske territorie så vil der fortsat fortsætte være massere masse af udfordring i syrien og irak vurderer ekspert de advare samtidig om at is ise ideologi vil bestå og at gruppe støtter vil udgøre en terrortrussel i vesten vest den forfærdelig sandhed er at is ise vil være lige så livsfarlig en oprørsgruppe og farligt et terrornetværk som det var da det minded minde om en stat siger mellemøstekspert nicholas nichola heras fellow ved center for a new american security til nyhedsbureau reut udfordring er nu at is ise vil blive et hævngerrigt spøgelse der vil forsøge at skab skabe kaos fortsætter fortsætte han aymenn jawad al tamimi der forsker forske i jihadisme ved tænketank middle east forum er enig han forvente at islamisk stat vil fortsætte med at benytte sig af selvmordsbombere sprængladninger og have såkaldt sove celler rundtom i verden efter kalifatets sammenbrud angribe angreb i europa vil fortsætte noget tid endnu jeg tro at sejr over is ise som et statslignende projekt vil svække gruppe tiltrækningskraft men is ise vil have støtter i lange lang tid endnu siger tamimi ifølge nyhedsbureau afp islamisk stat sidde fortsat fortsætte på område i irak og syrien men primært i grænseområdet mellem de to land lande ifølge det amerikansk militær har is ise omkring kriger tilbage i område ved grænse er gruppe dog allerede under presse pres af del dels syriske regeringsstyrker med opbakning fra rusland del dels arabisk og kurdiske styrker der få får støtte støt af usa selv hvis det område som vente bliver befri vil is ise ekstreme tolkning af islam bestå mener charlie winter han er seniorforsker ved international centrum center centre for the study of radicalisation and political violence jeg tro ikke på at hvis man bare bar fjerne islamisk stats territorie så forsvinde islamisk stats ideologi siger han til afp charlie winter mener at is ise fortsat fortsætte se sig selv som en succes fordi gruppe har formå at udråbe et kalifat og holde det køre i flere år her blev der eksempelvis både båd lave lav lavet mønte mønt og udstede passe pas ingen anden har gøre noget ligne lignende i ny tid det vil mener han have en effekt på global jihadisme i årevis islamisk stat udråbe sit kalifat i juni siden side dengang har gruppe miste knappe knap procent af sit territorie ifølge en talsmand for den amerikanskledede koalition der bekæmpe gruppe i syrien og irak" \\
		\midrule
		1 & 29851 & 0.983 & "et skridte skridt nærmerecenter fem millioner kommunale krone hjælpe hjælper nationalparkcenter på veje vej thy forlig forlige om thisted kommune budget for har hjælpe nationalpark thy et pænt pæn skridte skridt nær nærmere virkeliggørelsen af plan om at opføre et nationalparkcenter i vorupør i kommune budget for er der afsætte et beløbe beløb på fem millioner krone til nationalpark stor formidlingsprojekt trædesten til natur hvor et nytte ny nationalparkcenter bliver en af nationalpark ansøgning fremgå at de fem millioner krone skulle bruge til nationalparkcentret vi er godt godte god tilfreds med at penge er afsætte til trædestens projektet som dække hele hel område siger torben juul olsen formand for nationalpark bestyrelse budget i millionklassen trædesten til natur har et budget på millioner nationalparkcentret med et budget på millioner krone er en del dele af projektet med vished for yderligere yderlig fem millioner kommunale krone oveni de fem som allerede er tildele i forme form af værdi af en grunde grund og parkeringsarealer i vorupør konstatere torben juul olsen at nationalpark nu er meget tæt tætte på at være i mål måle med bestræbelse for at finansiere det komme nationalparkcenter inklusive de fem millioner kommunale budget krone har nationalpark nu millioner krone ud af byggesummen på mio krone ud over thisted kommune har nationalpark selv nordea fond og støtteforening nationalpark thy bidrage bidrag til beløbe beløb endnu mangel mangle nationalpark thy en udmelding fra en foreløbig ikke navngive fond som er ansøge om at støtte støt byggeri af nationalparkcentret med mio krone vi regne med at få fond tilkendegivelse senest i begyndelse af siger torben juul olsen mangel mangle halv million falde den positiv ud mangel mangle krone beløbe beløb vente rejse af støtteforening nationalpark thy som har bidrage bidrag med en million og foreløbig har indsamle krone af den halv million nationalpark har søgt søge kommune om forlængelse af dispensationen fra naturbeskyttelseslovens paragraf til byggeri af centre centrum center i vorupør dispensationen er oprindelig givet give for tre år" \\
		\midrule
		2 & 5507 & 0.993 & "det ske tirsdag november øst øse vildsunde vildsund færgekro fremtid er digital infomøde om landbonords digitale løsning bankohall bankohal nykøbing lotterispil nykøbing kirke festkoncert med musica ficta og bo holten nykøbing kirkecenter højskoleaften ved pastor leiff hvass fjordglimte fjordglimt kom og synge bio men mens vi levere leve lever med krop og sjæle sjæl thor ragnarok foredrage foredrag sære sans sanse jigsaw onsdag november morsø kirkehøjskole ole dybro om reformationsmaleren luca cranach aktivitetshus i sejerslev banko ansgarshjem kortspil hvidbjerg plejecenter banko sundhedscenter i nykøbing cafe for kræftramt og samvær med anden arr morsø lokalforening kb støberigård virkelyst underholdning ved dorthe græsborg det ottekantet forsamlingshus øster jølby valgtræffe valgtræf arr morsø folkeblad og radio limfjord superbrugs øster jølby km motionstur arrangere af morsø fodslaw bio men mens vi levere leve lever med krop og sjæle sjæl thor ragnarok jigsaw indlæg indlægge til liste list skal sendes pr maile mail til redaktion mf dk og der skal stå det ske i emnefelt deadline er mandag klokke kl uge før føre din dit arrangement torsdag november johan rii minde strikkecafe johan rii minde gå tur ture støberigård banko med mulighed for hjælp hjælpe støberigård knipling m m rotary park torsdagsklub ansgarshjem knipling dansk skaldyrcenter tang hav planter plante foredrage foredrag morsø teater den glade enke gæstespille gæstespil operettekompagniet arr morsø teaterkreds bio victoria abdulle fantasten a bad bede bade moms momse christma jigsaw indlæg indlægge til liste list skal sendes pr maile mail til redaktion mf dk og der skal stå det ske i emnefelt deadline er mandag klokke kl uge før føre din dit arrangement fredag november støberigård åben aktivitetscenter ansgarshjem synge sang støberigård gymnastik på stole stol johan rii minde banko bio fantasten victoria abdulle the treasure hunt a bad bede bade moms momse christma jigsaw indlæg indlægge til liste list skal sendes pr maile mail til redaktion mf dk og der skal stå det ske i emnefelt deadline er mandag klokke kl uge før føre din dit arrangement lørdag november skarregaard julemarked bio den utrolig historie om den kæmpestor pære my little pony film sikken et cirkus victoria og abdulle fantasten a bad bede bade moms momse christma jigsaw indlæg indlægge til liste list skal sendes pr maile mail til redaktion mf dk og der skal stå det ske i emnefelt deadline er mandag klokke kl uge før føre din dit arrangement søndag november skarregaard julemarked sundbyvej thy morse mor husflid afholde husflidsmesse bankohall bankohal nykøbing lotterispil bio operabio norma den utrolig historie om den kæmpestor pære my little pony film sikken et cirkus victoria og abdulle fantasten a bad bede bade moms momse christma jigsaw indlæg indlægge til liste list skal sendes pr maile mail til redaktion mf dk og der skal stå det ske i emnefelt deadline er mandag klokke kl uge før føre din dit arrangement mandag november støberigård åben aktivitetscenter johan rii minde gymnastik ansgarshjem motion på stole stol støberigård vi hækle m m rotary park kortklub støberisal seniorbridge sprogskole i frøslev nordmor byorkester spiller spille midtmorse midtmor sport på vag vagt i afghanistan foredrage foredrag med peter fyllgraff vil fritidscenter banko bio victoria og abdulle fantasten glasslot a bad bede bade moms momse christma indlæg indlægge til liste list skal sendes pr maile mail til redaktion mf dk og der skal stå det ske i emnefelt deadline er mandag klokke kl uge før føre din dit arrangement tirsdag november ansgarshjem vi nørkle støberigård åben aktivitetscenter støberigård oplæsning og snakke snak udfra bog om egon plejdrup støberigård højskoleforedrag ved hans ole vester støberigård patchwork m m aktivitetshus i sejerslev besøge besøg af præst anna holck ansgarshjem gudstjeneste dansk skaldyrcenter østerssafari johan rii minde gudstjeneste smørdalsvej ljørslev forberedelse til juleværksted mørkehistorier m m arrangere af morsø naturklub bankohall bankohal nykøbing lotterispil støberisal julearrangement bio fantasten victoria og abdulle foredrage foredrag inspiration fra natur a bad bede bade moms momse christma indlæg indlægge til liste list skal sendes pr maile mail til redaktion mf dk og der skal stå det ske i emnefelt deadline er mandag klokke kl uge før føre din dit arrangement" \\
		\midrule
		3 & 24775 & 0.952 & "årig model rådgive rådgiver trump washington det hvid hvide hus hu huse har udnævne den årig hope hicks til den indflydelsesrige poste post som kommunikationschef det bekræfte sarah huckabee sanders der er talsperson for det hvid hvide hus hu huse hicks har i de sen uge arbejde som midlertidig kommunikationschef for den amerikansk præsident donald trump men nu er stilling blive faste fast den tidlig tidligere model og pr konsulent er tæt tætte på trump familie og har været rådgive rådgiver for donald trumps datter ivanka hun er den tredje kommunikationschef i det hvid hvide hus hu huse i løb løbe af otte måned ritzau ap" \\
		\midrule
		4 & 12748 & 0.995 & "rh ne er et sikkert sikker vælge til julemadenden kraftige vin lade lader sig ikke slå ud af rødkål og julesul vi har testet flotte vin vine og er imponere de røde rh ne vin vine er et godt godte god julevalg vi ved det for vi har selv dem smage og det kan betale sig at læg lægge en hundredekroneseddel eller to eller flere oven i det vin koster koste til daglig for generel stige oplevelse med prise pris det kunne nordjyske smagehold konkludere efter flaske rh ne fra nord til syde syd der var safte saft og kraft og massere masse af kvalitet rh ne er selvfølgelig et vidt vid begribe begreb der er kæmpe forskel på om man drikke den rene syrah fra hermitage i nord eller en c tes du rh ne lang længe sydpå eller en ch teauneuf du pape hvor der måtte må indgå forskellig drue men de du dur dure alle sammen til vores vi traditionel julemad med deres gode god rene garvesyrer og en solid struktur c tes du rh ne village c tes du rh ne famille perrin coudoulet de beaucastel smv krone point lidt lide parfumeret duft dufte frisk friske bærduft fylde godt godte god i mund temmelig krydre lidt lide tørt bid bide til sidst fin balance dom r m jeanne les legantiers village bichel krone point temmelig lukke men åbner åbne sig pænt pæn når den få får lufte luft i glas kan evt dekanteres en smule rustik i duften fin frugt frugte i smag god balance røde bær bære le bouquet des garrigues vinspecialisten krone point ren og klare klar duft dufte finte fint fin bid bide frisk friske og umiddelbar vin cairanne dom oratoire st martin les douyes bichel krone point flotte flot tæt tætte farve koncentrere duft dufte med hint hin af chokolade tæt tætte vin med god eftersmag krydre lidt lide lakrids og et godt godte god bid bide af fin garvesyrer cornas cornas alain verset bichel krone point lædere læder og urter i duften intens og tør tø turde tørre vin med godt godte god integrerede garvesyrer gigondas syterres de bois neuf vin vin krone point alder ses se i farve der har få et brunt skære skær det fornemmes også i duften med noter af modne moden blommer afrundet vin der ikke holde holder til for meget eddike i rødkålene le primitif vin vin krone point duft dufte af røde bær bære kraftig robust vin med flotte flot struktur la r f rence vin vin krone point frisk friske bouquet af bær bære delikat bløde blød garvesyrer delas les reinages vinspecialisten krone point lidt lide lukke i duften så den måtte må gerne dekanteres fin balance ren frugt frugte tør tø turde tørre god mineralitet flotte flot vin dom cayron philpson krone point der er fornemmelse af alder i duften fin balance forholdsvis lette let le vin domain brusset les hauts de montmirail smv krone point kraftig vin dyb farve stort stor ekstrakt den kan virkelig bære julemaden chateauneuf du pape sabon le secret des sabon meny krone point fulde fuld fart farte på duften af brombær og modne moden blommer alder er tydelig i smag er der modne moden blommer lakrids og afrundede tørre tør tanniner godt godte god udvikle vin flotte flot vin domaine santa duc la crau ouest bichel krone point lædere læder i duften skal dekanteres for at åbne åben sig kraftig temmelig ung vin med mange muskler ch de beaucastel smv krone point duft dufte af sort kirsebær elegant bourquet flotte flot struktur perfekt balance delikat vin xavier vignon arcane v le pape supervin krone point intens bouquet brombær og mynte lækker læk lække vin i perfekt balance tør tø turde tørre med nogen mineralitet forbavsende ung olivier lafont meny krone point fyldig vin med flotte flot mørke mørk bær bære i duften den fylde hele hel mund nærmest fed fede men alligevel elegant her er mange muskler clos saint jean vinspecialisten krone point ren flotte flot bouquet lækker læk lække frugt frugte i smag der var også en fornemmelse af chokolade sabon prestige meny krone point lædere læder og mynte i duften tør tø turde tørre intens og krydre vin flotte garvesyrer elegant xavier vignon la r serve supervin krone point jeg var glad for stil fra den producent fin balanceret duft dufte i den ung unge vin alt var i balance elegant xavier vignon cuv e anonyme supervin krone point starte lidt lide lukke efter en snurretur i glas åbne den sig finte fint fin med modne moden bærnoter flotte flot veludviklet vin xavier vignon chateauneuf du pape supervin krone point ung vin med lidt lide urter og mynte i duften delikat vin med flotte flot afrundede garvesyrer fin balance sabon r serve meny krone point elegant vin dejlig duft dufte af mørke mørk bær bære god mineralitet ch de vaudieu philipson wine krone point lukke i duften ved opskænkning men den åbne flotte flot i glas give den en tur ture i en karaffel det åbner åbne også smag markere markeret garvesyrer der kan spille finte fint fin sammen med den fede fed made mad sabon les olivets meny krone point fin frugt frugte i næse elegant bouquet god fylde i smag ch fortia cuv e du baron bichel krone point kirsebær i duften lidt lide frugtsødme i smag en meget ung og frisk friske vin med godt godte god integrerede garvesyrer c te rotie delas seigneur de maugiron vinspecialisten krone point ung vin duft dufte af mørke mørk bær bære god frugt frugte og afrundede garvesyrer den er lige til at drikke drik og det er forbavsende med syrah vin i den alder balance mellem rødkålene den fede fed made mad og vin er garantere et hitte hit patrick jasmin smv krone point elegant afbalanceret duft dufte af brombær og kirsebær i mund er det en kraftig sag med massere masse af garvesyrer der skal luftes godt godte god for at blive afrundet meget ung vin der kan klare klar al den julemad den kombinere med hermitage delas domaine des tourettes vinspecialisten krone point flotte flot duft dufte med noter af kaffe lækker læk lække vin med brombær i smag fin balance mineralsk intens garvesyre der er perfekt integrere nicolas perrin smv krone point flotte flot vin delikat duft dufte der er en anelse animalsk fin balance i smag brombær kaffe og fornemmelse af træ crozes hermitage les croix bichel krone point her er der syrah for alle penge både båd i duft dufte og smage smag temmelig parfumeret næste rå i duften ung druepræget vin den kan man rolig roligt anvende sammen med al den tunge tung julemad der var stor spredning i karaktergivningen" \\
		\bottomrule
	\end{xltabular}
\newpage
\twocolumn
