\subsection{Distributions of the Corpus and Topics by Word Count}
In this section we will look at the word count distributions for the corpus and the topics shown in \autoref{fig:30TopicWords}, and describe some observations.

In \autoref{fig:corpus_wc_distribution} the word count distribution of the whole corpus can be seen.
Here the word count can be seen of the x-axis, and the number of documents that have these word counts are on the y-axis.
This gives an overall idea of the size of the documents after preprocessing.
Just by looking at the distribution, it is clear that there are a lot of documents of smaller sizes.
There is a clear spike from word counts of around 20-100, and this drops steadily until around a word count of 1000, where only few documents appear afterwards.
Statistics of this distribution can also be seen on the right side of the figure.
This tells us that, on average, the documents have a size of 256, and a median relatively close of 190.
It also confirms that the distribution is mainly in the range from 24-1182, which can be seen from the 1st and 99th percentile.
This is a good sign, since part of our preprocessing was meant to remove the empty and very short documents, to still have enough context to find topics.

\autoref{fig:topic_wc_distribution}

\begin{figure*}[h]
	\centering
	\includegraphics[width=\textwidth]{{"figures/Document_word distribution - corpus2017"}.pdf}
	\caption{The word count distribution of the corpus.}
	\label{fig:corpus_wc_distribution}
\end{figure*}

\begin{figure}[h]
	\centering
	\includegraphics[width=0.45\textwidth]{{"figures/Word count distribution per topic_corpus2017_final_model(30, 0.1, 0.1)(30, 0.1, 0.1)_topic 0-4"}.pdf}
	\caption{The word count distributions of the first 5 topics from the 30-topic model.}
	\label{fig:topic_wc_distribution}
\end{figure}