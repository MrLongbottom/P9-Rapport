\section{Introduction} 
\todo[inline]{should be twice as long as it is now}
%% Argue briefly for the relevance of the studied area/problem — start from general and end with your concrete problem.
A general challenge for search engines, no matter the environment, is how to acquire relevant search results from a given corpus.
While some search engines use simple algorithms, such as finding all articles including a given search term and sorting by date, this may not always give useful results.
Therefore we want to explore ways to improve search results while using information already available from a corpus.
\todo[inline]{jumping between labels and search}
Labeling documents in a corpus and using these labels to get relevant results could improve queries.
A popular way to label documents is through the use of topic modeling. \todo[inline]{Overvej om der er et bedre ord en "label", da det måske kan være misvisende.}
Topic modeling gives each document in a corpus a distribution of topics, which can be used for different applications.
More specifically, we look into \gls{lda} since it is currently the most used topic model, and is applicable for many situations\cite{lda}.
\todo[inline]{explain what topic modelling is and what it actually does}
Topic modeling does not rank documents in a search, which is why an additional part is necessary.
\todo[inline]{add two sentences to explain everything}
We have chosen to explore a popular way to rank documents, which is by using PageRank.\todo[inline]{add half a sentence explaining the principle}
PageRank is an algorithm originally used for ranking web pages in search engines, but it can easily be extended to other types of data, such as news articles.
A more specific implementation of PageRank is a cluster-based PageRank\cite{ClusterPageRank}.
This method uses similarities between sentences, clusters, and a corpus, to create better summaries of texts.
\todo[inline]{more! you’re mixing tanning and summaries here connection missing: pagerank does ranking.}
We believe the information propagation \todo[inline]{briefly explain information propagation} used in \cite{ClusterPageRank} between sentences can be applied to articles by clustering according to topics given by \gls{lda}.

Our goal is thus to improve news article search by using a combination of \gls{lda} and PageRank.
For our experiments we use the Nordjyske database of news articles as our dataset.
The idea is to use topic modeling to apply topics to documents, and cluster the documents with similar topics.
The cluster-based PageRank, which takes these clusters into account, is then used to find relevant documents given a search query.

%A combination of LDA along with cluster PageRank for generating relevant search results
\todo[inline]{Contributions}

The paper is organized as follows:
In \autoref{sec:related-works}, articles related to topic modeling and random walk are investigated.
\autoref{sec:dataset} describes the dataset used in the experiments.
In \autoref{sec:method}, the steps of our method are shown, and \autoref{sec:prepro}, \autoref{sec:language_model}, \autoref{sec:lda}, \autoref{sec:search_query}, and \autoref{sec:pagerank} describe these steps further.
In \autoref{sec:experiment}, the test combinations and hyperparameters are detailed and results are shown.
\autoref{sec:discussion} analyzes and discusses these results.
Finally, in \autoref{sec:conclusion} and \autoref{sec:future_work}, conclusions and future work are given.
