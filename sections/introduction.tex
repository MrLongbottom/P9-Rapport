\section{Introduction}
%% Argue briefly for the relevance of the studied area/problem — start from general and end with your concrete problem.
A general challenge for search engines, no matter the environment, is how to acquire relevant search results from a given corpus.
While some search engines use simple algorithms, such as finding all articles including a given search term and sorting by date, this may not always give useful results.
Therefore we want to explore ways to improve search results while using information already available from a corpus.

Labeling documents in a corpus and using these labels to get relevant results could improve queries.
A popular way to label documents is through the use of topic modeling. \todo[inline]{Overvej om der er et bedre ord en "label", da det måske kan være misvisende.}
Topic modeling gives each document in a corpus a distribution of topics, which can be used for different applications.
More specifically, we look into \gls{lda} since it is currently the most used topic model, and is applicable for many situations\cite{lda}.

Topic modeling does not rank documents in a search, which is why an additional part is necessary.
We have chosen to explore a popular way to rank documents, which is by using PageRank.
PageRank is an algorithm originally used for ranking web pages in search engines, but it can easily be extended to other types of data, such as news articles.
A more specific implementation of PageRank is a cluster-based PageRank\cite{ClusterPageRank}.
This method uses similarities between sentences, clusters, and a corpus, to create better summaries of texts.
We believe the information propagation used in \cite{ClusterPageRank} between sentences can be applied to articles by clustering according to topics given by \gls{lda}.

Our goal is thus to improve news article search by using a combination of \gls{lda} and PageRank.
For our experiments we use the Nordjyske database of news articles as our dataset.
The idea is to use topic modeling to apply topics to documents, and cluster the documents with similar topics.
The cluster-based PageRank, which takes these clusters into account, is then used to find relevant documents given a search query.

These questions also emphasize the contributions this paper brings.
We implement a method for constructing an adjacency matrix in order to run \gls{pr}, with no previous information given about connections between the nodes.
This is done by calculating the topic similarity between each document, using Jensen-Shannon distance, and using these similarities as the elements in the matrix.
We also suggest ways to measure the performance of an \gls{ir} method which has to find both specific documents and documents that share a specific topic.
Our system can be used to improve query results for search engines, by introducing a more abstract view of topics within a search query.

The paper is organized as follows:
In \autoref{sec:related-works}, articles related to topic modeling and random walk are investigated.
In \autoref{sec:method}, the theory behind LDA and clustering PageRank is detailed, and the pipeline is described.
\autoref{sec:experiment}
\autoref{sec:discussion}
\autoref{sec:conclusion} and \autoref{sec:future_work}
