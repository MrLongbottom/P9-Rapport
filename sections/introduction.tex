\section{Introduction} 
\vejleder[inline]{should be twice as long as it is now}
%% Argue briefly for the relevance of the studied area/problem — start from general and end with your concrete problem.

Many news articles are produced every day and the need for searching the news is becoming a more prominent and difficult task.
A general challenge for search engines, no matter the environment, is how to acquire relevant search results from a given corpus. 
While some engines use simple algorithms, such as finding all articles that include a given search term and sorting by date, this may not always give useful results.
Ranking articles can be done multiple ways, where PageRank is commonly used\cite{google_pagerank2006}.
PageRank\cite{pagerank_1999} is a ranking algorithm, which is based on the random walk process.
The principle is to let a number of surfers walk between nodes in a graph, where each surfer takes one step in each iteration.
After an appropriate number of steps, the number of surfers on each node will be the ranking of the given node, where more surfers give a higher rank.
The edges in the graph can be based on many interactions, such as hyperlinks, similarities, genres, etc.
In this paper, we want to explore ways to improve search results while using information already available from a corpus.
Documents in a corpus can contain relevant information, which could yield improvements for search engines.
A popular way to extract this information is through the use of topic modeling.
Topic modeling is a method within machine learning and \gls{NLP} which builds a model that can find abstract topics within a corpus of text documents.
A topic model assigns a distribution of topics to each document in a corpus, which can be used for different applications.
More specifically, we look into \gls{lda} since it is currently the most used topic model, and is applicable for many situations\cite{lda}.
\gls{lda} uses Dirichlet distributions to calculate the probability of a word belonging to a given topic and a document containing a given topic.
Topic modeling does not rank documents, which is why an additional part is necessary if we want to search the documents.
We intend to investigate whether topic information can be used to improve search engines.
We have chosen to explore using PageRank along with topic information, since it is a popular way to rank documents.
PageRank was originally used for ranking web pages in search engines, but it can easily be extended to other types of data, such as news articles.
A more specific implementation of PageRank is a cluster-based PageRank\cite{ClusterPageRank}.
This method uses similarities between sentences, clusters, and a corpus, to create better summaries of texts.
It utilizes these three similarity functions to introduce new information between sentences in a document and combines these into one score for a sentence.

We want to investigate whether the cluster level information of topics can be combined into a PageRank to rank documents based on their topics.
Topic themes yield important information between sentences in \cite{ClusterPageRank} and we believe that this concept can be applied to articles by clustering according to topics given by \gls{lda}.

Our goal is thus to improve news article search by using a combination of \gls{lda} and PageRank.
For our experiments we use the Nordjyske database of news articles as our dataset.
The idea is to use topic modeling to apply topics to documents, and cluster the documents with similar topics.
The cluster-based PageRank, which takes these clusters into account, is then used to find relevant documents given a search query.

These questions also emphasize the contributions this paper brings.
We implement a method for constructing an adjacency matrix in order to run \gls{pr}, with no previous information given about connections between the nodes.
This is done by calculating the topic similarity between each document, using Jensen-Shannon distance, and using these similarities as the elements in the matrix.
We also suggest ways to measure the performance of an \gls{ir} method which has to find both specific documents and documents that share a specific topic.
Our system can be used to improve query results for search engines, by introducing a more abstract view of topics within a search query.

The paper is organized as follows:
In \autoref{sec:related-works}, articles related to topic modeling and random walk are investigated.
In \autoref{sec:method}, the theory behind LDA and clustering PageRank is detailed, and the pipeline is described.
\autoref{sec:experiment}
\autoref{sec:discussion}
\autoref{sec:conclusion} and \autoref{sec:future_work}
