\section{Search Query}\label{sec:query}
%One of the challenges of preprocessing queries is that they usually contain very few words.
%Therefore, there is a huge possibility of information loss during preprocessing and conversion, as there is not much content or context to be learned from.
%Our method converts each word in the query into a topic distribution, making the query have the same format as a document. \todo[inline]{Detaljer specifikt hvordan denne distribution laves, når det er færdigimplementeret.}
%This enables us to compare the query to other documents, and therefore do a random walk based on the query.
%This is an important process, since we need to compare different models against how well they retrieve the articles.
%The danger of this method is that a query only containing a couple of words, will likely have a very skewed topic distribution.

We use two techniques for generating the queries used for evaluating our search algorithm, namely document queries and topic queries.
The idea behind these two methods is to evaluate the specificity and generality of the \gls{ir} methods, respectively.
For our experiment we will be generating 80 queries for each length from 1 to 4 for the two query types, making a total of 320 document queries and 320 topic queries ($\sim$1$\%$ of the entire corpus).
The construction of these queries is described in the following subsections:

\subsubsection{Document query}\label{subsec:query_gen_doc}
To generate a document query, we randomly select a document from the corpus and a query length $ql$ between $ql_{min}$ and $ql_{max}$.
Document queries are all based on different documents.
From this selection, we apply \gls{tf-idf} on the selected document to find the most informative words.
Our query is then the $ql$ highest ranked words within the given document.
The goal is to check whether or not the \gls{ir} method is able to find the specific document based on a search query generated from that document.

\subsubsection{Topic query}\label{subsec:query_gen_top}
The topic query is slightly different in comparison to document query in that it focuses on the topics rather than a specific document.
To construct a topic query, firstly we select a random topic $t$ and a query length $ql$ between $ql_{min}$ and $ql_{max}$.
Topic queries can be based on the same topic but will have a high variance in the actual words used.
We then sample $ql$ documents based on the document-topic distributions $\theta$ provided by the \gls{lda} model.
Though $\theta$ is normally one topic distribution for each document, we create a document distribution for each topic by column normalizing the matrix.
This is necessary, because when observing a specific topic column in the document-topic distribution matrix $\theta$, the values do not work as probabilities, since they do not add up to 1.
This column normalization will make each topic become a distribution of all documents' topic distribution values for the specific topic.
This is essentially a topic-document distribution matrix.
When sampling the documents, we only sample within the topic-document distribution where the topic is set to $t$.
From these sampled documents, we extract one word from each document, based on the \gls{tf-idf} score, and add these to the query.
If an extracted word is in the query already, we skip it and add the next highest \gls{tf-idf} scored word to the query.
In that way, we ensure to get unique words in the query.
The goal, of generating this query, is to check whether the \gls{ir} method can find similar documents based on a search query generated from a topic.

%\subsection{Query Handling}
%Searching for a given document in a whole data set of documents can be a challenge. 
%In this setting, we propose to use the topic distributions of each document for searching.
%The search process consists of multiple steps before we are able to use a query within the PageRank algorithm. 
%We start with a search query consisting of words which are stemmed.
%For each word in the query, we check whether our model knows the word.
%If it is known, we use the $\beta$ distribution given by the model for the given word. 
%If it is not known, we let the model predict the topic distribution for the given word.
%We take the average of these distributions and calculate the Jenson-Shannon distance between this distribution and all other document-topic distributions to figure out which articles are the most similar to the given query.
%The final vector has the length of the number of documents and has a value from zero to one in all cells, which describes how similar a query is to the given documents based on the topics.
%We then use the resulting vector as our personalization vector for a cluster-based random walk, when searching/ranking articles.
 
