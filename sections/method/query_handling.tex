\section{Search Query}
We have identified three steps when working with a search query:
\begin{itemize}
	\item Generation
	\item Expansion
	\item Handling
\end{itemize}

We will describe each step in the following sections.

\subsection{Query Generation}


\subsection{Query Expansion}
Query expansion is a method which adds more words to a given query to add more context to the search process.
\citet{yang2009topic} describes a query expansion method, called window query expansion.
Queries are expanded based on the neighboring words in the document set.
Specifically, for each word in the query, we count the frequency of each word appearing before and after the given word in the corpus.
We then add top $m$ most frequent words to the query.
An example of this could be the query: 
\begin{quote}
	Query = "Football"
\end{quote}
As an example the most frequent words before and after "Football" in our document set could be:
\begin{itemize}
	\item Coach - 15
	\item Match - 10
	\item Ball - 5
\end{itemize}
A resulting expanded query with an $m$ = 2 would then be:
\begin{quote}
	Query = "Football Coach Match"
\end{quote}



\subsection{Query Handling}
Searching for a given document in a whole data set of documents can be a challenge. 
In this setting, we propose to use the topic distributions of each document for searching.
The search process consists of multiple steps before we are able to use a query within the PageRank algorithm. 
We start with a search query consisting of words which are stemmed.
For each word in the query, we check whether our model knows the word.
If it is known, we use the $\beta$ distribution given by the model for the given word. 
If it is not known, we let the model predict the topic distribution for the given word.
We take the average of these distributions and calculate the Jenson-Shannon distance between this distribution and all other document-topic distributions to figure out which articles are the most similar to the given query.
The final vector has the length of the number of documents and has a value from zero to one in all cells, which describes how similar a query is to the given documents based on the topics.
We then use the resulting vector as our personalization vector for a cluster-based random walk, when searching/ranking articles.
 
