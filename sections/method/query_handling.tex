\section{Search Query}
We have identified three steps when working with a search query:
\begin{itemize}
	\item Generation
	\item Expansion
	\item Handling
\end{itemize}

We will describe each step in the following sections.


\subsection{Query Generation}


\subsection{Query Expansion}


\subsection{Query Handling}
Searching for a given document in a whole data set of documents can be a challenge. 
In this setting, we propose to use the topic distributions of each document for searching.
The search process consists of multiple steps before we are able to use a query within the PageRank algorithm. 
We start with a search query consisting of words which are stemmed.
For each word in the query, we check whether our model knows the word.
If it is known, we use the $\beta$ distribution given by the model for the given word. 
If it is not known, we let the model predict the topic distribution for the given word.
We take the average of these distributions and calculate the Jenson-Shannon distance between this distribution and all other document-topic distributions to figure out which articles are the most similar to the given query.
The final vector has the length of the number of documents and has a value from zero to one in all cells, which describes how similar a query is to the given documents based on the topics.
We then use the resulting vector as our bias for a cluster-based random walk, when searching/ranking articles.
 
