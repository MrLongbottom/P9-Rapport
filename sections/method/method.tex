\section{\gls{ir} Methods}\label{sec:ir_methods}

In this section we will be introducing the \gls{ir} methods that we will be testing.
We have chosen these methods to evaluate different ways of performing a \acrfull{ir} task.

The goal of each method is to produce a document-score vector with one score for each document in the corpus, indicating the relevance of the document compared to the query.
These scores can then be sorted in descending order to produce a ranking of the documents.

Some of the methods calculate scores for a single word at a time rather than a whole query.
In that case, we treat their score is an estimate of the probability a document $d$ generating a specific word $w$.
To find the probability of $d$ generating a given query $q$, we take the product of the word probabilities for each word in the query $w \in q$, as the document would have to generate each word, as shown in \autoref{eq:query_prob}.

\begin{equation}\label{eq:query_prob}
	P(q|d) = \prod_{w \in q} P(w|d)
\end{equation}

\section{\acrlong{lda}}\label{sec:lda}
\Gls{lda} is a probabilistic topic model method. 
The objective of topic modeling is to infer topics (collections of words) in a document set.
The result consists of a topic-word distribution matrix $\beta$, which for each topic gives a distribution of words belonging to said topic, and a document-topic distribution matrix $\theta$ which for each document gives a distribution of topics to which the document belongs to.

\citeauthor{lda} \cite{lda} introduce \gls{lda}, which has since become a staple within topic modeling.
\gls{lda} works under the assumption that documents are generated from a specific generative process, and tries to reverse engineer this process.
This generative process assumes that documents are random mixtures of latent topics and that each topic is a distribution over all the words in the corpus.

\begin{definition}\label{def:topic}
	\textit{A topic is a distribution over words in a corpus}
\end{definition}

This process generates $K$ topics and $D$ documents containing $N_{d}$ words, where $d$ is a document in $D$.
The generative process has two Dirichlet distributions $Dir(\alpha)$ for the document-topic relation and $Dir(\eta)$ for the topic-word relation.

\begin{figure}[h]
	\centering
	\includegraphics[width=0.5\textwidth]{figures/Smoothed_LDA.jpg}
	\caption{Plate notation for \gls{lda}. Boxes symbolize repeated processes, shaded elements are observed information. Image is from \url{https://en.wikipedia.org/wiki/Latent_Dirichlet_allocation}}
	\label{fig:lda}
\end{figure}
\vejleder[inline]{beskriv inference og learning af modelen i appendix}
From these Dirichlet distributions, we can sample two multinomial distributions: document-topic $\theta$ and topic-word $\beta$.
These Dirichlet distributions are tuned with the hyperparameters $\alpha$ and $\eta$, which adjust the entropy of the sampled distributions.
An $\alpha$ value near 1 causes each document to be distributed over almost all topics, while an $\alpha$ value near 0 causes each document to be distributed over only a few topics.
Similarly $\eta$ will adjust how many words each topic contains.
This also has the added consequence that a high $\alpha$ will make documents appear more similar, and a high $\eta$ will make topics appear more similar.

Using $\theta$ and $\beta$, we can sample concrete topics $Z$ from documents, and concrete words $W$ from topics.
\autoref{fig:lda} gives an overview of the assumed generative process.

\todo[inline]{mention the limitations of lda.}

\section{Language model}
In \cite{yang2009topic}, they describe various combinations of \gls{lda} and other models. 
The language model they describe is similar to a query likelihood model, which generates a probability of how likely a given document $d$ produces a given query $q$.
To calculate this probability, they need to find the likelihood for each word in the query by using function 

$$ P(w|d) = \frac{N_d}{N_d + \lambda} \cdot \frac{tf(w,d)}{N_d} + (1 - \frac{N_d}{N_d + \lambda}) \cdot \frac{tf(w,D)}{N_D} $$
where $N_d$ is the number of word tokens in $d$ and $tf(w,d)$ is the word frequency of $w$ in $d$. $\lambda$ is a Dirichlet smoothing factor and is set to the average document length.
When $D$ is used, it is referring the document collection.
We multiply the word probabilities together in order to get the probability for $q$.

$$ P(q|d) = \prod_{w \in q} P(w|d) $$
 
The \gls{lda} model gets poor retrieval performance when it not used in combination with another model \cite{yang2009topic}.
They combine the \gls{lda} model and the language model to get information about the topic and word correlation between $q$ and $d$.

\section{PageRank}\label{sec:pagerank}
We construct a graph over our documents, where edges between documents $d_i$ and $d_j$ are based on the similarity between them $sim(d_i, d_j)$.

This allows us to use PageRank on the graph to search for similar documents.
On top of this, we can add a personalization vector, to guide the search in a specific direction.
We can base this personalization vector on a search query to find documents similar to the query.

To construct the adjacency matrix for the graph, we apply a similarity function on each pair of document-topic distributions $\theta_{d_i}, \theta_{d_j}$, from our topic model.

We use Jenson-Shannon distance as our similarity function, which is a measure of the distance between probability distributions\cite{jensen-shannon2003}\cite{jensen-shannondis2003}.
It is calculated using the square root of the Jenson-Shannon divergence between two probability distributions and gives a score from 0 to 1.
However since Jenson-Shannon calculates difference rather than similarity, a low value means more similarity, making our similarity function:
$$sim(d_1, d_2) = 1 - JS(\theta_{d_1}, \theta_{d_2})$$


%\subsection{Clustering PageRank}
%
%In \cite{ClusterPageRank}, they describe the \gls{Cluster-CMRW} which is a new version of the random walk model. 
%They improve upon this model by incorporating information from clusters. 
%The clusters are found by employing three different clustering algorithms.
%They create a new transition probability matrix where the cluster information is added by combining three similarity functions based on the information levels present in the data.
%They describe three similarity levels between:
%\begin{itemize}
%    \item Sentence to Sentence - $f(v_i \rightarrow v_j)$
%    \item Sentence to Topic Cluster - $\omega(v_i, clus(v_i))$
%    \item Topic Cluster to Document Set - $\pi(clus(v_i))$
%\end{itemize}
%where $v_i$ is the given sentence and $clus(v_i)$ is the cluster that $v_i$ belongs to.
%
%We want to use this method to incorporate the information given by the topic distributions within a \gls{Cluster-CMRW}. 
%We first create three similarity functions to incorporate topic cluster level information into a new adjacency matrix.
%\begin{itemize}
%    \item Document to Document - $f(d_i \rightarrow d_j)$
%    \item Document to Topic Cluster - $\omega(d_i,clus(d_i))$
%    \item Topic Cluster to Document Set - $\pi(clus(d_i))$
%\end{itemize}
%
%\noindent
%Where $d$ is a document, $clus(d_i)$ is a topic cluster of document $d_i$.
%Notice that a document $d$ can contain multiple topics, as opposed to normal clustering where each element is generally assumed to only be part of one cluster.
%
%\subsection*{Document to Document Similarity}
%As in \autoref{sec:pagerank} we use Jensen-Shannon distance as our base for document-to-document similarity:
%$$ f(d_i \rightarrow d_j) = 1 - JS(d_i, d_j)$$ 
%
%\subsection*{Document to Topic Cluster Similarity}
%We define the similarity between a document and a topic cluster as the document-topic probability distribution between them.
%$$ \omega(d_i,clus(d_i)) = \theta_{d_i,clus(d_i)}$$
%where $\theta$ is the document-topic distribution matrix.
%
%\subsection*{Topic Cluster to Document Set Similarity}
%We define the similarity between a topic $t$ and the whole document set as:
%$$ \pi(clus(d_i)) = \frac{\sum_{d}^{D} \theta_{clus(d_i)}}{D} $$
%where $D$ is the number of documents in the document set.
%
%
%To create the adjacency matrix, \cite{ClusterPageRank} linearly combine the different similarity functions, in which we intend to do the same.
%Formally, the adjacency matrix is calculated with the following function.
%$$ f(d_i \rightarrow d_j | clus(d_i), clus(d_j)) $$
%which is evaluated to 
%
%\begin{align*}
%&f(d_i \rightarrow d_j | clus(d_i), clus(d_j)) = \\
%&f(d_i, d_j) \cdot (\lambda \cdot \pi(clus(d_i))) \cdot \omega(d_i, clus(d_i)) \\ 
%&+ (1-\lambda) \cdot \pi(clus(d_j)) \cdot \omega(d_j, clus(d_j))
%\end{align*}
%
%where $\lambda$ is a weight between $[0,1]$ that controls the relative contribution of the two clusters.


\subsection{\acrlong{tf-idf}}
\Gls{tf-idf} is a well known \gls{ir} method which is used to find the most important words within text.
As the name suggests, this methods aims to balance the two measures: term frequency and inverse document frequency.
For our case, terms are the words left in the documents after the preprocessing phase.
For each term in the query $t \in q$, a tf-idf score will be calculated for each document in the corpus $d \in D$.
Thus each document will have one score for each term in he query.


Term frequency describes how often a word is used in the specific document that is being evaluated.
Inverse Document Frequency describes how many documents in the corpus includes the given term, with a higher percentage of documents producing a lower value.
The overall goal of \gls{tf-idf}, can be summarized as a measure of how often used and unique a term is for a given document. Words that are often used in the given document, but rarely used in the corpus will have the highest \gls{tf-idf} scores.

The formula for \gls{tf-idf} can be seen in \autoref{eq:tfidf}
\begin{equation}\label{eq:tfidf}
	\text{tf-idf}(w, d, D) = \text{tf}(w, d) \cdot \text{idf}(w, D)
\end{equation}
where tf$(t, d)$ is the number of times term $t$ is in document $d$ and idf$(t, D)$ is the inverse document frequency of $t$ in $D$.

\subsection{\acrlong{bm25}}
\todo[inline]{describe goal / purpose / idea}
\gls{bm25} is a bag of words retrieval function, which is similar to \gls{tf-idf}, since it uses the inverse document frequency.
The formula for \gls{bm25} can be seen in \autoref{eq:bm25}\cite{bm25}.
\begin{equation}\label{eq:bm25}
	\text{bm25}(d, q) = \sum_{i=1}^{n}\text{idf}(q_i) \cdot \frac{\text{tf}(q_i, d) \cdot (k_1 + 1)}{\text{tf}(q_i, d) + k_1 \cdot (1 - b + b \cdot \frac{|d|}{avgdl})}
\end{equation}
where tf$(q_i)$ is the number of times the term $q_i$ appears in $d$.
$|d|$ is the number of words in d. 
$avgdl$ is the average document length from the corpus.
$b$ and $k_1$ are both hyperparameters, which are set to $0.75$ and $1.5$, respectively.

\subsection{Combining \gls{ir} Methods}
Each of the \gls{ir} methods described in this section has their own strengths and weaknesses.
We combine multiple of these methods, to see if they are able to draw upon each others strengths and cover each others weaknesses.
We do with the hope of a combination of methods being able to produce better results than each of the methods separately.

As with \citet{yang2009topic}, we combine multiple \gls{ir} methods together, by normalizing their document-score vectors and then either summing or multiplying them together element-wise.
This process is visualized in \todo{ref figure}.

\todo[inline]{insert figure.}

Both of these two options are viable, and have their own benefits and drawbacks.
Summation allows documents to be ranked fairly high even if one of the combined methods produce a low score.
With multiplication a document with average scores for each method can have a better rank than one with mostly good scores but a really low score from one of the combined \gls{ir} methods.
