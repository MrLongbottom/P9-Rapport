\section{Method}\label{sec:method}
This section describes our method on an abstract level.
Each phase of the method is described later on in more detail. 
We start with a data set, in our case consisting of articles from the media group Nordjyske. 
Their primary focus is to maintain a variety of local newspapers within the North Jutland region of Denmark. 
The data ranges from 2017, where a total of $\sim$~63.000 articles have been extracted from their database.

\subsection*{Step 1: Preprocessing Phase}
This phase applies different \gls{NLP} methods, such as stemming and removing stop words, to simplify the data set and remove redundant information.
Details of this phase are given in \autoref{sec:prepro}.
After finishing this phase, we are left with $\sim$~32.000 articles, which will be used in the next step.

\subsection*{Step 2: LDA}
We train a \acrfull{lda} on the data set to generate topics based on the content of articles within the data set. 
We describe the investigation and selection of hyper parameters in \autoref{subsec:lda}. 
After the model has been trained, we have a document-topic distribution matrix $\theta$ and a topic-word distribution matrix $\beta$.
$\theta$ in the next phase.

\subsection*{Step 3: Query Preprocessing}
A search query, consisting of words, is stemmed and turned into a list of words.
For each word, we generate a topic distribution based on $\beta$ or prediction of the model.
We use this distribution, as a personalization vector for some of our models.


\subsection*{Step 4: Evaluation of models}
In this phase, we evaluate different combination of models using the evaluation metric described in \todo{ref evaluation metric section}. 
The combinations and results can be seen in section \todo{ref combination section } and \todo{ref result section }, respectively.

%\tikzstyle{process} = [rectangle, rounded corners, minimum width=2cm, minimum height=1cm,text centered, draw=black, fill=gray!50]
\tikzstyle{decision} = [diamond, minimum width=3cm, minimum height=1cm, text centered, draw=black, fill=green!30]

\begin{figure}[ht]
    \centering
    \begin{tikzpicture}[node distance=2cm]
    %\draw[step=1cm,gray,very thin] (-8,-8) grid (8,8);
	\node (Dataset) [process] {Nordjyske Dataset};
	\node (Cleaning)[process, below of=Dataset] {Preprocessing Phase};
	\node (Training) [process, below of=Cleaning] {Train IR Methods};
	\node (Cluster PR) [process, below of=Training] {Evaluate Models};
	\node (Query) at (-4.5, -6) [process] {Query Generation};
	\node (Result) [process, below of=Cluster PR] {Results};
	\draw [->, very thick] (Dataset) edge (Cleaning); 
	\draw [->, very thick] (Cleaning) edge (Training);
	\draw [->, very thick] (Training) edge (Cluster PR);
	\draw [->, very thick] (Cluster PR) edge (Result);
	\draw [->, very thick] (Query) edge (Cluster PR);
\end{tikzpicture}
	\caption{The method visualized as a flowchart, where a dataset consisting of articles is processed into a list of ranked results.}
    \label{fig:process_figure}
\end{figure}

%The pipeline of our framework is divided into five phases which are displayed in \autoref{fig:pipeline}.
