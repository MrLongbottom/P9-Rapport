\section{Pipeline}
This section describes our framework on an abstract level.
Each of these phases are described later on in more detail. 
We start this pipe with a data set consisting of articles from the media group Nordjyske. Their primary focus is to maintain a variety of local newspapers within the North Jutland region of Denmark. 
The data ranges from 2017-2019, where a total of $\sim$270.000 articles have been extracted from their database.

\subsection*{Step 1: Preprocessing Phase}
This phase applies different \gls{NLP} methods, such as stemming and removing stop words, to simplify the data set and remove redundant information.
After finishing this phase, we are left with $\sim$130.000 articles, which will be used in the phase.

\subsection*{Step 2: LDA}
We train an \acrfull{lda} on the data set to generate topic clusters based on the content of the articles. 
We describe the investigation and selection of hyper parameters in \autoref{subsec:lda}. 
After the model has been trained, we have a document-topic distribution matrix $\theta$ and a topic-word distribution matrix $\beta$.
$\theta$ will be used as clusters in the next phase.


\subsection*{Step 3: Query Preprocessing}
A search query, consisting of words, is stemmed and turned into a list of words.
For each word, we generate a topic distribution based on $\beta$ or prediction of the model.
We combine these distributions and use them to bias our cluster-based random walk, when searching/ranking for articles.


\subsection*{Step 4: Cluster-based Random Walk}
In this phase, we use cluster-based random walk, described in \autoref{sec:cluster_pagerank}, to rank the articles based clusters given in the \gls{lda} phase.
Before we are able to rank the articles, we need to process a search query first to figure out which articles to focus on.
The final product is a prioritized list of articles based on the search query.


%\tikzstyle{process} = [rectangle, rounded corners, minimum width=2cm, minimum height=1cm,text centered, draw=black, fill=gray!50]

\begin{figure}[h]
    \centering
    \begin{tikzpicture}[node distance=1.5cm]
    %\draw[step=1cm,gray,very thin] (-8,-8) grid (8,8);
	\node (Dataset) [process] {Article dataset};
	\node (Cleaning)[process, below of=Dataset] {Preprocessing Phase};
	\node (Training) [process, below of=Cleaning] {LDA};
	\node (Cluster PR) [process, below of=Training] {Clustering PageRank};
	\node (Query) at (-3.5, -4.5) [process] {Query Processing};
	\node (Result) [process, below of=Cluster PR] {Results};
	\draw [->, very thick] (Dataset) edge (Cleaning); 
	\draw [->, very thick] (Cleaning) edge (Training);
	\draw [->, very thick] (Training) edge (Cluster PR);
	\draw [->, very thick] (Cluster PR) edge (Result);
	\draw [->, very thick] (Query) edge (Cluster PR);
\end{tikzpicture}
	\caption{Pipeline}
    \label{fig:pipeline}
\end{figure}
%The pipeline of our framework is divided into five phases which are displayed in \autoref{fig:pipeline}. 
